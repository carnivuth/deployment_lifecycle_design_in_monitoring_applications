\documentclass[12pt,a4paper,twoside,openright]{book}

% instead of paragraph indentation on first line and no spacing
% makes no indentation and spacing
\usepackage{parskip}

\usepackage[utf8]{inputenc}
\usepackage[english]{babel}
\usepackage[T1]{fontenc}

\usepackage{style/isi_style_lt}

\usepackage{amsmath,amsfonts,amssymb,amsthm}
\usepackage{caption}
\usepackage[usenames]{color}
\usepackage{enumerate}
\usepackage{fancyhdr}
\usepackage{fancyvrb}
\usepackage{float}
\usepackage{booktabs}
\usepackage{indentfirst}
\usepackage{listings}
\usepackage{marvosym}
\usepackage{multicol}
\usepackage{sectsty}
\usepackage{tocloft}
\usepackage{microtype}
\usepackage[table]{xcolor}
\usepackage{url}
\usepackage{hyperref}
\usepackage[toc]{appendix}

\usepackage{dsfont}
\usepackage{comment}
\usepackage{multirow}
\usepackage{hhline}
\usepackage{adjustbox}
\usepackage{tkz-tab}
\usepackage{pgfplotstable}
\usepackage{pgfplots}
\usepackage{subcaption}

\usepackage{tablefootnote}

\usepackage{graphicx}
\usepackage{tikz}
\usepackage{forest}
\usetikzlibrary{trees,positioning,shapes,shadows,arrows.meta}

\definecolor{bar1}{HTML}{FFEF9F}
\definecolor{bar2}{HTML}{FFCFD2}
\definecolor{bar3}{HTML}{CFBAF0}
\definecolor{bar4}{HTML}{90DBF4}
\definecolor{line1}{HTML}{BC4749}
\definecolor{line2}{HTML}{168AAD}
\definecolor{line3}{HTML}{7B2CBF}

\definecolor{radar1}{HTML}{E3B505}
\definecolor{radar2}{HTML}{34B2E4}
\definecolor{radar3}{HTML}{065381}
\definecolor{radar4}{HTML}{59A96A}
\definecolor{radar5}{HTML}{E34856}
\definecolor{radar6}{HTML}{FE912A}

\newcommand*{\paramsbox}[2]{{\small \colorbox{white!#1!red}{\color{black} #2}}}
\newcommand*{\carbonbox}[2]{{\small \colorbox{white!#1!green}{\color{black} #2}}}

\hypersetup{%
	pdfpagemode={UseOutlines},
	bookmarksopen,
	pdfstartview={FitH},
	colorlinks,
	linkcolor={black},
	citecolor={black},
	urlcolor={black}
}

\AtBeginDocument{%
	\renewcommand{\contentsname}{Indice}%Indice
	\renewcommand\tablename{Tabella}%Tabella
	\renewcommand{\listtablename}{Elenco delle tabelle}%Elenco delle tabelle
	\renewcommand\figurename{Figura}%Figura
	\renewcommand{\listfigurename}{Elenco delle figure}%Elenco delle figure
	\renewcommand{\lstlistingname}{Codice}%Listato
	\renewcommand{\chaptername}{Capitolo}%Capitolo
	\renewcommand{\refname}{Riferimenti}%Riferimenti
	\renewcommand{\bibname}{Bibliografia}%Bibliografia
}

\definecolor{dkgreen}{rgb}{0,0.6,0}
\definecolor{gray}{rgb}{0.5,0.5,0.5}
\definecolor{mauve}{rgb}{0.58,0,0.82}

\lstset{
  frame=single,
  captionpos=b,
  language=Java,
  aboveskip=3mm,
  belowskip=3mm,
  showstringspaces=false,
  columns=flexible,
  basicstyle={\small\ttfamily},
  numbers=none,
  numberstyle=\tiny\color{gray},
  keywordstyle=\color{blue},
  commentstyle=\color{dkgreen},
  stringstyle=\color{mauve},
  breaklines=true,
  breakatwhitespace=true,
  tabsize=3
}

\makeatletter
\def\cleardoublepage{
	\clearpage\if@twoside \ifodd\c@page\else
	\hbox{}
	\thispagestyle{empty}
	\newpage
	\if@twocolumn\hbox{}\newpage\fi\fi\fi
}

\makeatother

\setlength{\textwidth}{14cm}
\setlength{\textheight}{21cm}
\setlength{\footskip}{3cm}

\setlength{\hoffset}{0pt}
\setlength{\voffset}{0pt}

\setlength{\oddsidemargin}{1cm}
\setlength{\evensidemargin}{1cm}

\universita{Alma Mater Studiorum -- Università di Bologna}

\scuola{Scuola di Ingegneria}

\corsodilaurea{Corso di Laurea Magistrale in Ingegneria Informatica}

\titolo{Sistemi per la gestione scalabile del software life-cycle in applicativi di monitoraggio}

\materia{laboratorio di amministrazione di sistemi}

\laureando{Matteo Longhi}

\relatore[\footnotesize Chiar.mo Prof. Ing.\normalsize]{Marco Prandini}
%\correlatoreA[\footnotesize Prof.\normalsize]{}
%\correlatoreB[\footnotesize Dott. Ing.\normalsize]{}

\sessione{Seconda} % 'Prima', 'Seconda', 'Terza'

\annoaccademico{2024 -- 2025}

\parolechiave %devono essere 5 keywordsrolechiave,
{Provisioning}
{scalabilità}
{monitoraggio}
{automazione}
{software life-cycle}


\dedica{\emph{ai ragazzi del Network operation center dei laboratori Marconi e ai gorilla che mi hanno supportato in questo percorso}}

\makeindex

\begin{document}

\frontmatter

\maketitle

\chapter*{Sommario}
\markboth{Sommario}{Sommario}

Lo sviluppo software per risultare efficace necessita di una adeguata strategia di manutenzione e monitoraggio delle istanze in produzione per prevenire situazioni di software erosion. Il mio elaborato di tesi punta a mostrare quanto è stato svolto presso i Laboratori Marconi per quanto riguarda il software life-cycle del sistema di monitoraggio aziendale: Sanet.

Nel elaborato verranno affrontate problematiche quali:

\begin{itemize}
  \item Riorganizzazione dell'architettura di produzione del software
  \item Progettazione delle procedure di update del software
  \item Pianificazione ed esecuzione delle operazioni di migrazione delle istanze di produzione
\end{itemize}

L'obbiettivo del progetto che ho condotto presso i Laboratori Marconi e' stato allineare le installazioni di Sanet a una unica architettura di produzione che fosse in grado di coprire tutte le varie esigenze di business in termini di servizi erogati. L'architettura in questione aveva come requisito quello di essere replicabile in maniera programmatica.

Le sopracitate problematiche verranno esplorate con un occhio dedicato alla scalabilità e malleabilità delle operazioni e al mantenimento dei processi produttivi aziendali, con l'obbiettivo di rendere il Network operation center capace di gestire le operazioni regolari di manutenzione del software

Il progetto ha comportato un ampia fase di analisi e studio dei workflow aziendali tra cui:

\begin{itemize}
  \item Workflow operativi del network operation center, intercettazione e gestione degli allarmi generati da sanet e procedure di escalation
  \item Workflow operativi sistemistici legati alla manutenzione di Sanet e utilizzo come pulizia dei log e aggiunta a monitoraggio degli apparati
  \item Workflow di sviluppo di integrazioni e componenti di sanet sia da parte dei team sistemistici che dell'area di sviluppo
\end{itemize}

E stata posta sotto analisi l'architettura del software e di tutte le componenti aggiunte per estendere e/o garantire funzionalità avanzate necessarie per il contesto di produzione

Successivamente sono stati presi in analisi lo stato delle produzioni di sanet, identificate le situazioni critiche, i casi limite e le particolarità delle singole istanze, questo per avere coscienza di cosa sia necessario nell'architettura di produzione finale e definire il minimo insieme che contenesse tutte le features richieste per erogare il servizio di monitoraggio.

Ha seguito la progettazione e sviluppo della nuova architettura e del conseguente processo di migrazione delle istanze in produzione alla nuova architettura. Con la conseguente produzione della documentazione operativa e di sviluppo per la manutenzione e utilizzo del progetto.


\newpage

\tableofcontents

\newpage

\listoffigures

\mainmatter

\pagestyle{fancy}
\fancyhead[LO]{\nouppercase{\rightmark}}
\fancyhead[RE]{\nouppercase{\leftmark}}
\fancyhead[LE,RO]{\thepage}
\fancyfoot{}


\chapter{Introduzione, background teorico e tecnologico}

La manutenzione e il monitoraggio di un software in produzione e fondamentale per prevenire fenomeni di software erosion, gestire il software life-cycle nelle grandi installazioni e' sempre molto costoso, specialmente se il core business non e lo sviluppo software ma il servizio erogato tramite esso.

Il problema si presenta in maniera ancora maggiore quando la separazione dei contesti dovuta all'espansione del parco clienti porta a un crescere sempre maggiore del numero di istanze del software in produzione che erogano il servizio, gli effetti più comuni della software erosion sono:

\begin{itemize}
    \item dipendenze software non aggiornate che portano a incompatibilità applicative con standard dell'industria
    \item degradazione delle performance dovuta al riempimento dei dischi
    \item sistemi insicuri, dovuti al degrado di protocolli di cifratura utilizzati per le comunicazioni
\end{itemize}

Queste problematiche possono diventare causa di forti rallentamenti al workflow aziendale, specialmente se ignorati per molto tempo, impedendo agli utenti amministratori l'utilizzo del software dato che questi si vedono costretti a impiegare le loro risorse in operazioni di manutenzione del software che spesso e volentieri non sono risolutive del problema.

Un esempio può essere l'installazione manuale di dipendenze applicative, rese obsolete da aggiornamenti dell'ambiente di esecuzione ma necessarie all'applicazione per funzionare risultando in ambienti non solo instabili ma anche insicuri.

La software erosion incide anche sullo sviluppo del software in produzione, dato che l'aggiornamento di istanze "dimenticate" può richiedere troppe risorse invogliando cosi all'implementazione di workaround pur di soddisfare esigenze di servizio. In queste situazioni l'istanza in produzione e soggetta a una "morsa a tenaglia" dove il ciclo di rilascio del software si deve scontrare con integrazioni applicative sviluppate dagli utenti sistemisti che sfruttano API "non convenzionali" per arricchire le funzionalità del sistema stesso.

Il problema risulta accentuato se lo sviluppo non prevede la produzione e l'utilizzo di artefatti per la messa in produzione del software, in questo caso le infrastrutture in produzione sono alla merce di procedure di installazione manuali che portano a una frammentazione degli ambienti in produzione. La creazione di artefatti, specialmente per sistemi software complessi, non deve prevedere solo la messa in funzione delle componenti applicative sviluppate a doc ma anche di tutti i possibili servizi di cui l'applicazione necessita come:

\begin{itemize}
    \item servizi di data storage (databases,demoni cache)
    \item servizi web
    \item demoni per l'esecuzione di manutenzioni programmate
\end{itemize}

Per affrontare il problema sono nati i container, ambienti che sfruttano la funzionalità del kernel linux namespaces per isolare processi applicativi su differenti livelli. Questi sistemi offrono spesso approcci dichiarativi al problema consentendo allo sviluppatore la riproducibilità dell'ambiente di esecuzione e testing. Inoltre consentono di uniformare l'interfaccia fra sviluppo e operazioni di devops e rendere lo sviluppo consapevole dell'ambiente di produzione già in fase di progettazione.

I container offrono la possibilità allo sviluppo di progettare il funzionamento di features applicative che necessitano di un determinato supporto runtime dall'ambiente di produzione, per esempio assicurarsi della raggiungibilità via rete dei componenti dell'applicazione, creare ambienti di test in agilità simulando interazioni fra vari nodi in rete, testare integrazioni con altri ambienti prima di aggiornare la produzione

\chapter{Analisi dei requisiti}
%Requisiti di progetto e obbiettivo finale richiesto dalla missione di business

L'attività svolta presso i laboratori Marconi aveva come obbiettivo finale rendere i membri del network operation center team (NOC in breve) capaci di svolgere le operazioni di manutenzione delle installazioni del software di monitoraggio aziendale Sanet, in particolare:

\begin{itemize}
\item svolgere in autonomia operazioni di aggiornamento distribuito di sanet
\item programmare e organizzare attività di installazione di istanze di sanet
\item svolgere monitoraggio delle performance e conseguenti attività di manutenzione come pulizia di log applicativi, bilanciamenti di carico, aggiunta di nodi a monitoraggio
\end{itemize}

Questo con l'obbiettivo di poter successivamente spostare la responsabilità della gestione delle istanze in produzione al team NOC, e evitare che i membri dei singoli team di sistemisti si debbano occupare di tale attività, liberando risorse da dedicare a task sistemistici richiesti dai clienti.

E' stato inoltre richiesto che fossero revisionate le documentazioni tecniche riguardanti le attività di monitoraggio svolte presso i vari clienti, in modo da uniformare la base documentale e renderne più semplice l'accesso in maniera programmatica.

I requisti richiesti hanno come scopo ultimo quello di ridurre la quantità di ore uomo complessive spese in operazioni di manutenzione di Sanet, ridurre l'entropia delle installazioni in produzione accentrandone la gestione a un unico team


\chapter{Analisi del problema di business}
%Ampia descrizione delle problematiche operative (workflow aziendali non più manutenibili) e di sviluppo (mantenimento dell'applicativo, automazione delle operazioni di ciclo di vita del software) overview di ciò che già e stato fatto e come viene gestito il software in produzione

Il progetto ha previsto un'ampia fase di analisi dove sono stati documentati

\begin{itemize}
  \item{I processi aziendali portati avanti dai vari membri dei team}
  \item{I software coinvolti in tali processi sia proprietari che non}
  \item{Lo stato delle istanze di tali software in produzione}
\end{itemize}

L'obbiettivo di questa fase di analisi è stato quello di raccogliere informazioni sullo stato dell'installato in funzione delle operazioni compiute dal team NOC per soddisfare la mission aziendale, in particolare ci si è concentrati sul comprendere le motivazioni di determinate scelte progettuali in fase di messa in produzione del software di monitoraggio sanet.

\newpage
\subsection{Analisi dell'applicativo}
% Analisi del sistema di monitoraggio aziendale, con particolare riguardo agli aspetti interessanti per il deployment
Il business principale dei laboratori Marconi consiste nel monitoraggio di apparati e servizi applicativi, per erogare tale servizio e non dipendere da software esterno l'azienda ha deciso di sviluppare la sua soluzione di monitoraggio proprietaria: \verb|sanet|.

Dal punto di vista architetturale Sanet è un sistema distribuito basato su simple network message protocol (\verb|SNMP|), composto da processi demoni di varia natura tra cui:

\begin{itemize}
  \item{Demoni per lo storage dei dati}
  \item{Demoni per il recupero dei dati stessi}
  \item{Demoni per la gestione delle notifiche}
\end{itemize}

\begin{figure}[H]
    \centering
    \includegraphics[width=1\linewidth]{build/sanet_architecture.png}
    \caption{Architettura di sanet}
    \label{fig:enter-label}
\end{figure}

% architettura applicativa
L'architettura si compone di un demone principale sanetd che svolge i seguenti compiti:

\begin{itemize}
  \item{Gestione del flusso dati in ingresso e della ripartizione e scheduling dei task da eseguire per effettuare il monitoraggio}
  \item{Salvataggio di log applicativi e stato dei controlli su storage stabile}
  \item{Gestione delle timelines generate dai task di monitoraggio}
  \item{Gestione del flusso in ingresso per il demone di notifica}
\end{itemize}

% datagroups
I task generati dal demone principale vengono denominati \verb|datagroups|, un datagroup è la composizione di tre elementi principali:

\begin{itemize}
  \item \verb|datasource| sorgente dei dati, specifica le modalità con cui i dati devono essere reperiti, per esempio possono essere specificate OID SNMP, files, risorse web
  \item \verb|condition| controllo da effettuare sui datasource interessati, le condition sono la sorgente degli allarmi generati
  \item \verb|timegraph| contengono informazioni sullo storico del determinato datasource, per esempio le statistiche di banda consumata per una data interfaccia di rete
\end{itemize}

% conditions
Le condition di un datagroup possono assumere 4 stati diversi:

\begin{figure}[H]
    \centering
    \includegraphics[height=0.6\linewidth]{build/condition_status.png}
    \caption{Stati di una condition}
    \label{fig:enter-label}
\end{figure}

La transizione da uno stato all'altro avviene subito dopo l'esecuzione di una condition da parte di un poller, in particolare nel momento in cui l'espressione booleana della condition risulta falsa essa effettua una transizione nello stato \verb|failing|, al raggiungimento di un numero di tentativi falliti configurabile il controllo passa in stato \verb|down|, se il poller non riesce a valutare la condizione per esempio per irraggiungibilità del nodo o per fallimento di servizi necessari la condition transita in stato \verb|uncheckable|.

Il linguaggio di configurazione consente di specificare regole di dipendenza tra vari check per mezzo del parametro \verb|dependson|, una data condition che è in \verb|dependson| di un altra non viene verificata se le condition da cui dipende non si trovano in stato \verb|up|, in caso questo non sia verificato la condition passa in stato \verb|uncheckable|

\begin{figure}[H]
    \centering
    \includegraphics[height=0.6\linewidth]{build/dependson_management.png}
    \caption{Gestione di condition in dependson}
    \label{fig:enter-label}
\end{figure}

% tipologie di oggetti monitorabili
Sanet suddivide gli oggetti da sottoporre al monitoraggio in macro categorie note nel contesto del monitoraggio di apparati di rete,il generico elemento monitorabile viene modellato per mezzo del concetto di \verb|node| che rappresenta un generico asset IT, il monitoraggio di un \verb|node| viene specializzato per mezzo della seguente classificazione:

\begin{itemize}
  \item storage: supporti di memoria di vario genere sia fisici che virtuali
  \item service: servizi di qualunque entità, generalmente processi demoni
  \item interface: Interfacce di rete del nodo monitorato
  \item device: sensori ed entità affini
\end{itemize}

Per questi oggetti vengono previsti datasource e condition di default come lo stato dello storage o il traffico sulle interfacce.

% generazione e instradamento di allarmi
Nel momento in cui una condition transita nello stato \verb|down| viene generato un allarme, questo viene processato dal demone \verb|entables| che determina la corretta destinazione dell'allarme in base a tabelle e catene di regole, in particolare sono previste 2 tabelle principali

\begin{itemize}
  \item tabella di mangle: utilizzata per arricchire l'allarme di metadati per il processing da parte di sistemi esterni
  \item tabella di filter: questa tabella è quella dove il pacchetto viene instradato al corretto destinatario
\end{itemize}

Le tabelle sono composte da catene di regole, nel momento in cui un allarme viene processato da \verb|entables| questo scorre tutte le catene definite e se una data regola nella catena fa match con gli attributi dell'allarme vengono applicate le azioni di quella determinata regola e si procede alla regola successiva

\begin{figure}[H]
    \centering
    \includegraphics[height=0.6\linewidth]{build/entables_processing.png}
    \caption{Funzionamento di entables}
    \label{fig:enter-label}
\end{figure}

Un allarme può essere consegnato a destinazioni differenti, le più utilizzate sono la notifica attraverso \verb|SMTP| (email) oppure la consegna al sottosistema \verb|nocview|

% nocview comunicazione
Un allarme che fa match con la regola di instradamento \verb|store-alarms| viene destinato al sistema di notifica centrale dei laboratori: \verb|nocview|, la consegna dell'allarme avviene attraverso un modello di comunicazione polling, dove \verb|entables| salva l'allarme all'interno di un apposita tabella su database e \verb|nocview| esegue richieste http periodiche per recuperare gli allarmi della data tabella.
Gli allarmi raccolti dalle varie istanze vengono presentati ai membri del team NOC per mezzo di una singola interfaccia web suddivisi per istanza di sanet da cui sono stati estratti.

\begin{figure}[H]
    \centering
    \includegraphics[width=0.6\linewidth]{build/sanet_nocview_comunication.png}
    \caption{Interazione sanet nocview}
    \label{fig:enter-label}
\end{figure}

% scheduling delle conditions
Le condition vengono verificate dal sistema applicando uno schema a polling, il demone centrale inserisce nella coda le condition che devono essere verificate e i processi poller consumano le condition in coda e inseriscono i risultati che vengono raccolti dal processo centrale che aggiorna lo stato interno delle conditions. Le code vengono implementate per mezzo del key value store redis

Durante l'esecuzione della condition il poller recupera i valori di tutti i datasource coinvolti nella condition stessa, esegue l'espressione booleana della condition, aggiorna i \verb|timegraph| con i rispettivi valori e restituisce tutto al demone centrale per mezzo delle code di redis. In caso il controllo preveda l'invio di un pacchetto ICMP per verificare la connettività viene coinvolto i demone \verb|pingerd| che si occupa di effettuare le comunicazioni di rete

\begin{figure}[H]
    \centering
    \includegraphics[width=0.6\linewidth]{build/sanetd_pollers_interaction.png}
    \caption{Interazione sanetd pollers}
    \label{fig:enter-label}
\end{figure}

% linguaggio di configurazione
I nodi da monitorare e i rispettivi \verb|datagroup,storage,service,interface,device| vengono definiti per mezzo di un apposito linguaggio di configurazione, un esempio è il seguente

\begin{lstlisting}
node my.important.server
    agent main-agent
    description "Critical service for production, needs to be monitored"
    icon linux
    datagroup important datagroup
      condition important condition
      expr "2>1"
      exit
    exit
    storage rootfs distinguisher /
    exit
    interface ethernet distinguisher eth0
    exit
exit
\end{lstlisting}

Per poter distinguere particolari istanze di interfacce e storage si fa uso del parametro \verb|distinguisher| che viene utilizzato in fase di scansione della tabella SNMP che contiene informazioni sulle interfacce e sugli storage per identificare i dati interessati.

Il linguaggio di configurazione dei nodi consente inoltre di definire porzioni di configurazione cosiddetti \verb|templates| che consentono di raggruppare datagroup comuni per il monitoraggio di nodi in un unica configurazione e applicare una forma di ereditarietà, per esempio se si vogliono monitorare 10 macchine linux un possibile template può essere il seguente:

\begin{lstlisting}
node-template server-linux
    source site
    description "Server Linux"
    is-switch false
    is-router false
    datagroup clock-hr-tz
    exit
    datagroup cpu-hr
    exit
    datagroup defgw-mac-change
    exit
    datagroup icmp-reachability
    exit
    datagroup loadavg-threshold-ucd
    exit
    datagroup nonidle-ucd
    exit
    datagroup process-vlock-absence
    exit
    datagroup ram-usage-ucd
    exit
    datagroup reboot-hr
    exit
    datagroup snmp-status
    exit
    datagroup storage-all-linux
    exit
    datagroup swap-ucd
    exit
exit
\end{lstlisting}

Dove si definisce il monitoraggio dei servizi base di una macchina linux insieme alle componenti essenziali per le statistiche
I template possono contenere \verb|datagroups,condition,storages,interfaces,services,devices|,essi sono a loro volta classificati in base alla risorsa di cui sono template:

\begin{itemize}
  \item \verb|node-templates| templates da applicare al nodo monitorato, possono contenere tutti gli altri template
  \item \verb|service-templates| templates per singolo servizio applicativo
  \item \verb|storage-templates| templates per lo storage
  \item \verb|interface-templates| templates per le interfacce
  \item \verb|device-templates| templates per il monitoraggio di sensori generici
  \item \verb|datagroup-templates| templates per singolo datagroup, possono essere contenuti in tutti gli altri template e sono i maggiormente utilizzati
\end{itemize}

I template più comuni come quello di cui sopra vengono mantenuti in una raccolta comune a tutte le installazioni detta \verb|library|, essa viene importata all'interno delle istanze di sanet e i template al suo interno vengono personalizzati per mezzo di quanto necessario

\begin{figure}[H]
    \centering
    \includegraphics[width=1\linewidth]{build/sanet_library.png}
    \caption{Sanet library}
    \label{fig:enter-label}
\end{figure}

% linguaggio di espressione delle condition
Le espressioni all'interno delle condition fanno uso di un altro linguaggio denominato \verb|sanet standard expression language|, pensato per la manipolazione dei valori di ritorno delle OID SNMP, i controlli fanno uso di funzioni implementate in tale linguaggio, se per esempio si vuole monitorare la presenza di un servizio all'interno di un nodo la condition si mostrerà come segue

\begin{lstlisting}
condition process_presence
  expr "(1.3.6.1.2.1.1.1.0@) != '' and isDefined('\$swrunindex', catch_timeouts=False) == True"
exit
\end{lstlisting}

In ottica di estensibilità il linguaggio consente anche l'esecuzione di binari e script custom, per il monitoraggio di funzionalità non previste dalle funzioni builtin del linguaggio.

% tech stack di sviluppo
Dal punto di vista di sviluppo sanet è formato da un insieme di moduli python che fanno uso estensivo del framework django per la gestione della web-ui e della persistenza a database,

% agenti remoti
Per monitorare i nodi di una data rete è necessario che i processi di sanet possano raggiungere gli apparati interessati, questo in produzione non è sempre possibile per una serie di motivi quali partizionamento della rete per suddivisione dei compiti e regole firewall che non consentono il traffico dalla macchina di monitoraggio, inoltre in alcune installazioni geograficamente estese il monitoraggio risulta inefficiente in quanto le distanze rendono le misure di RTT esagerate e i timeout troppo stringenti, per affrontare il problema sanet offre la possibilità di utilizzare un poller remoto

\begin{figure}[H]
    \centering
    \includegraphics[width=0.3\linewidth]{build/remote_poller_architecture.png}
    \caption{Architettura del poller remoto}
    \label{fig:enter-label} \end{figure}

Il poller remoto è pensato come un servizio stateless che comunica con il demone sanetd per mezzo della istanza di redis del demone sanetd, ottiene la lista di condition da eseguire e restituisce al demone principale i risultati, notare che la comunicazione è iniziata dal poller remoto


\begin{figure}[H]
    \centering
    \includegraphics[width=0.8\linewidth]{build/remote_poller_sanetd_interaction.png}
    \caption{Interazione sanetd poller remoto}
    \label{fig:enter-label}
\end{figure}

Sanet modella le entità che eseguono le condition come \verb|agenti|, ogni agente è affidato a un poller e ogni poller corrisponde a un set di processi pesanti unix e a un sottoinsieme di thread python dedicati al particolare agente:


\begin{figure}[H]
    \centering
    \includegraphics[width=0.8\linewidth]{build/agents_vs_pollers.png}
    \caption{Relazione tra poller e agenti}
    \label{fig:enter-label}
\end{figure}

% sum up
Sanet si mostra come un complesso sistema distribuito in grado di adattarsi al contesto di produzione e fornire agli amministratori della rete un buon potere espressivo nel definire le operazioni richieste per il monitoraggio.

\newpage
\subsection{Analisi del contesto aziendale}

% struttura area NMS
L'area network management and security (NMS) dei Laboratori è strutturata in Team, ogni team fa capo a un sistemista senior e risponde di un sottoinsieme di clienti. Per ogni cliente dei Laboratori viene predisposta un istanza del sistema Sanet in una macchina dedicata presso l'infrastruttura IT del cliente, l'amministrazione è affidata ai sistemisti che ne hanno in carico la commessa.

% gestione richieste di monitoraggio
Le richieste di messa a monitoraggio dei clienti vengono svolte dai membri del team o dai membri del Network Operation Center (NOC) a seconda della complessità dell'operazione.

\begin{figure}[H]
    \centering
    \includegraphics[height=0.6\linewidth]{build/team_schema.png}
    \caption{Schema operatività team sistemistici}
    \label{fig:enter-label}
\end{figure}

Le richieste di monitoraggio possono spaziare tra una vasta gamma di casistiche tra cui:

\begin{itemize}
  \item{Aggiunta a monitoraggio di elementi nuovi elementi noti}
  \item{Monitoraggio di applicativi di varia natura (database,servizi web, posta)}
  \item{Monitoraggio di infrastrutture hardware non note}
  \item{Monitoraggio di nodi mobili}
\end{itemize}

In caso la richiesta preveda dello sviluppo software essa viene presa in carico dai membri del team che valutano e progettano un componente software che, interfacciandosi con sanet fornisce il servizio richiesto.

% workflow di update
I membri del team sono anche responsabili per la manutenzione e aggiornamento delle istanze di cui sono amministratori, il workflow di update prevede che l'area NAD notifichi il team della necessità di un aggiornamento delle istanze in produzione.

\begin{figure}[H]
    \centering
    \includegraphics[width=1\linewidth]{build/update_workflow.png}
    \caption{Workflow di update}
    \label{fig:enter-label}
\end{figure}

Il team inoltre determina come l'architettura di sanet venga distribuita sull'infrastruttura del cliente, considerando le risorse che il cliente mette a disposizione per il monitoraggio.

% workflow team NOC

Il team NOC gestisce due workflow aziendali:

\begin{itemize}
  \item{Incident response: intercettazione delle problematiche dell'infrastruttura del cliente e esecuzione di procedure di presa in carico e analisi del problema}
  \item{Service request: soddisfacimento di una specifica richiesta da parte del cliente}
\end{itemize}

Per entrambi questi workflow il team NOC fa affidamento a una base documentale manutenuta dai team sistemistici dove vengono scritte le procedure i referenti e le principali problematiche di un tale componente infrastrutturale del cliente

\newpage
\subsection{Analisi stato della produzione}

% "artefatti" prodotti dal area NAD (stendiamo un velo pietoso....)
L'area NAD fornisce sanet sotto forma di repository subversion contenenti i sorgenti con una guida all'installazione e aggiornamento del software, l'applicativo è pensato per essere eseguito all'interno di un virtual environment di python dove vengono installati i moduli necessari, il sistema a runtime deve anche rendere disponibili i servizi attivi necessari al funzionamento come redis e postgres. Questi requisiti vengono soddisfatti dalla infrastruttura a runtime fornita dal team,

\begin{figure}[H]
    \centering
    \includegraphics[width=1\linewidth]{build/sanet_production_simple.png}
    \caption{Architettura sanet in produzione}
    \label{fig:enter-label}
\end{figure}

% sanet e bind locale
Il sistema sanet, data la necessità di effettuare richieste di rete in maniera periodica genera un flusso costante di richieste DNS, questo porta a una degradazione delle prestazioni della rete e a statistiche di monitoraggio viziate come valori di RTT più alti e traffico in uscita dagli apparati di rete aumentato. Per migliorare la situazione e rendere il sistema più fault tolerant si introduce un server DNS locale al processo poller, questo si comporta da slave del DNS master del cliente, scarica le zone all'avvio e effettua la risoluzione dei nomi localmente al sistema sanet azzerando il traffico di rete.


\begin{figure}[H]
    \centering
    \includegraphics[width=1\linewidth]{build/sanet_bind.png}
    \caption{Sanet e bind locale}
    \label{fig:enter-label}
\end{figure}

In questo modo sanet è in grado di continuare a erogare il servizio anche in caso di guasti al servizio DNS.

% sanet e le estensioni per il monitoraggio
Le integrazioni per monitoraggi si presentano sotto molte forme, quelle maggiormente riscontrate in produzione sono:

\begin{itemize}
  \item{Script bash o python eseguiti direttamente dal poller}
  \item{Script bash o python eseguiti periodicamente per mezzo del demone cron}
\end{itemize}

In caso di esecuzione per mezzo di cron l'integrazione e sanet interagiscono in maniera disaccoppiata nel tempo e nello spazio, lo script comunica con sanet per mezzo del filesystem, scrivendo file all'interno della macchina di monitoraggio o per mezzo del demone redis, aggiungendo chiavi al database

\begin{figure}[H]
    \centering
    \includegraphics[width=1\linewidth]{build/sanet_integrations.png}
    \caption{Sanet e integrazioni sistemistiche}
    \label{fig:enter-label}
\end{figure}

Le competenze delle integrazioni possono variare da controlli infrastrutturali che monitorano lo stato di particolari servizi come tunnel vpn, backup job di virtualizzatori, stato di tablespaces oracle oppure monitorare particolari API di sistemi complessi come VMware.

% sanet e rancid
Una delle principali funzionalità che offre sanet è quella di monitorare apparati di rete che espongono agenti SNMP, uno dei compiti assegnati alla macchina di monitoraggio dai sistemisti è anche quello di eseguire il backup periodico degli apparati per mezzo del tool \verb|really awesome cisco config differ| (\verb|rancid|).

Il programma rancid sfrutta il programma \verb|clogin| per connettersi agli apparati, esegue dunque uno script \verb|perl| per recuperare le informazioni di configurazione dell'apparato in forma testuale e ne salva le modifiche per mezzo di un sistema di controllo di versione.

Per evitare di configurare rancid manualmente i sistemisti hanno sviluppato un integrazione con sanet che data una struttura dati che fa uso dei tag di sanet recupera dinamicamente la lista di nodi che devono essere sottoposti alla procedura di backup di rancid

\begin{figure}[H]
    \centering
    \includegraphics[width=1\linewidth]{build/sanet_rancid.png}
    \caption{Integrazione sanet rancid}
    \label{fig:enter-label}
\end{figure}

% sanet di backup
Per garantire il servizio anche in caso di guasti fisici all'infrastruttura che ospita il sistema in alcune installazioni grandi di sanet viene predisposta una seconda istanza parallela, questa esegue in maniera parallela alla precedente e il database di sanet viene mantenuto sincronizzato con la produzione principale, in caso di guasti l'istanza secondaria sostituisce la principale nell'invio delle notifiche al sistema \verb|nocview|

\begin{figure}[H]
    \centering
    \includegraphics[width=0.6\linewidth]{build/sanet_backup.png}
    \caption{Ridondanza nelle installazioni di sanet}
    \label{fig:enter-label}
\end{figure}

\newpage
\subsection{Analisi delle problematiche insorte}

% frammentazione dell'installato
Delegare l'ingegnerizzazione delle istanze ai singoli team ha si ridotto il carico di lavoro dell'area NAD ma ha anche comportato con il tempo una situazione di frammentazione delle installazioni di sanet, le feature necessarie al monitoraggio sono implementate in maniera diversa a seconda della conoscenza a disposizione dei team di sanet e delle tecnologie utilizzabili.

% analisi delle integrazioni
Le integrazioni sviluppate da sistemisti per aggiungere funzionalità a sanet sono riconducibili tutte alla seguente struttura:

\begin{itemize}
  \item{Raccolta dati da una fonte \(X\)}
  \item{Parsing dei dati raccolti da un formato \(\alpha\) a un formato \(\beta\)}
  \item{Comunicazione con il sistema sanet}
\end{itemize}

\begin{figure}[H]
    \centering
    \includegraphics[width=1\linewidth]{build/sanet_integration_structure.png}
    \caption{Struttura delle integrazioni}
    \label{fig:enter-label}
\end{figure}

Le prime due fasi sono le più problematiche, la prima comporta spesso la comunicazione con sistemi che nel caso migliore parlano protocolli open ma a versioni non più supportate come ad esempio vecchie versioni di TLS, nel peggior caso parlano protocolli proprietari instabili e a volte anche binari.

Quanto detto viene esacerbato ancora di più nella seconda fase, il parsing delle informazioni raccolte e infatti strettamente dipendente dal formato dei dati recuperati che è da considerarsi instabile e soggetto a mutazioni frequenti.

Per Semplificare l'implementazione i sistemisti fanno spesso uso di librerie e o programmi esterni che si occupano di risolvere la complessità della comunicazione e o del parsing, in caso di script python questi possono anche condividere il virtual environment di sanet, rendendo impossibile la manutenzione delle dipendenze del software.

La fase di comunicazione con sanet risulta comunque problematica in quanto non viene definita nessuna API "ufficiale" di comunicazione con l'applicazione stessa, di conseguenza le integrazioni comunicano con sanet per mezzo di sistemi complessi e non noti come può essere l'esecuzione di script della command line interface di sanet, modifica di file nel filesystem, manipolazione delle chiavi in redis e cosi via.

Le integrazioni inoltre non sono sottoposte a procedure di versioning o code review e non è prevista nessuna procedura di deployment al di fuori degli ambienti dove vengono sviluppati, di fatto si perde la conoscenza dello stato dell'installato, rendendo estremamente difficile qualunque operazione di manutenzione necessaria all'area NAD per aggiornare il software.

% problemi della messa a monitoraggio
I team sistemistici fanno uso intensivo dei template per accelerare l'aggiunta a monitoraggio di un oggetto, tuttavia agiscono in maniera indipendente e le configurazioni di nodi simili risultano estremamente diverse, per esempio dato un server linux che fornisce un servizio web la configurazione può fare uso di template in composizione o in ereditarietà

\begin{figure}[H]
    \centering
    \includegraphics[width=1\linewidth]{build/sanet_config_comparison.png}
    \caption{Due diverse configurazioni ma con lo stesso obbiettivo}
    \label{fig:enter-label}
\end{figure}

Le due possibilità di configurazione danno luogo a scenari diversi, nel caso di utilizzo dell'ereditarietà si ha si un'alta capacita espressiva in fase di progettazione del monitoraggio ma una minore malleabilità nel caso di scenari non previsti, per esempio in caso di un nodo che abbia multipli server web oppure che offra funzionalità ulteriori come il servizio radius, se invece si opta per la composizione si guadagna in malleabilità ma si delega la scelta dei template di cui comporre il nodo sempre in fase di messa a monitoraggio incrementando cosi il carico di lavoro.

Il sintomo di queste problematiche lo si ha nello stato dell'installato che verte in una condizione di estrema software erosion e frammentazione, rendendo impossibile il transito della gestione al team NOC che non è a conoscenza delle scelte progettuali effettuate e delle motivazioni che hanno portato a compiere tali scelte.

\chapter{Progettazione della soluzione}
%Progettazione della architettura di produzione del software applicativo, gestione di tutti i componenti critici per la corretta erogazione dei servizi offerti e progettazione del sistema di software management per la gestione del life-cycle, progettazione di tutte le operazioni necessarie per la migrazione alla nuova architettura

Dato il contesto descritto sopra uno dei sotto obbiettivi risultava essere l'allineamento dell'infrastruttura di sanet a una sola architettura comune, che semplificasse l'installazione, e l'aggiunta di componenti a sanet, i requisiti di questa infrastruttura sono stati:

\begin{itemize}
\item{riproducibilità della stessa}
\item{versioning dell'infrastruttura}
\item{possibilità di gestione e monitoraggio centralizzati}
\end{itemize}
% Il problema del deployment
Il problema principale riscontrato in fase di analisi e quello legato alla gestione del deployment da parte dei due team coinvolti NAD e NMS, nel workflow mostrato la progettazione del deployment e delegata ai team sistemistici, impedendo all'area NAD di conoscere lo stato delle istanze in produzione, inoltre l'assenza di artefatti prodotti fa si che fra la progettazione del deployment e le effettive operazioni di messa in produzione non esista un interfaccia chiara che costringa le due parti a un workflow condiviso.

Risultava quindi necessario progettare un artefatto e la conseguente procedura di build che dati i sorgenti di sanet producesse un oggetto installabile, ma che allo stesso tempo lasciasse ai team sistemistici la possibilità di modificare l'ambiente di esecuzione in caso di necessità per rispondere tempestivamente alle richieste dei clienti.

\newpage
\subsection{Progettazione dell'artefatto: Il container}

La necessità di pacchettizzare sia le dipendenze applicative che demoni come redis e postgres ha fatto si che il team optasse per una soluzione basata su container.

\begin{figure}[H]
    \centering
    \includegraphics[width=1\linewidth]{build/container.png}
    \caption{container come artefatto}
    \label{fig:enter-label}
\end{figure}

% LXC come soluzione a container
Per la creazione del container il team ha optato per la tecnologia LXC\cite{LXC} che offre un set di API di basso livello per la creazione di ambienti runtime isolati sfruttando le possibilità offerte dalla funzionalità namespaces del kernel linux. La tecnologia LXC non fa assunzioni sulla natura del container e consente la creazione di ambienti virtuali completi di init system dove all'interno possono risiedere processi demoni di varia natura.

% struttura del container LXC
Nel container LXC vengono predisposti tutti i moduli di sanet e viene creato il virtual environment di python necessario per l'esecuzione, vengono inoltre installati tutti i demoni necessari al funzionamento di sanet, e inoltre presente il software rancid per il backup periodico delle configurazioni degli apparati e la sua integrazione con sanet

% gestione dei dati
Sanet e rancid producono una mole di dati proporzionale al numero di datagroup monitorati di cui viene mantenuto uno storico di un anno, in un installazione medio piccola questo comporta \(\simeq 1000 GB\) di occupazione di storage, questi dati inoltre devono essere sottoposti a backup e sopravvivere a una distruzione del container, per garantirlo viene sfruttata la tecnologia bind mount offerta da LXC\cite{LXC}
I dati prodotti da sanet e rancid vengono esportati sulla macchina host per mezzo di bind-mount.

\begin{figure}[H]
    \centering
    \includegraphics[width=1\linewidth]{build/container_data_management.png}
    \caption{gestione dei dati e bind mounts}
    \label{fig:enter-label}
\end{figure}

% gestione della network
Il container di sanet fa parte di una rete privata con indirizzamento \verb|169.254.254.x/24|, il demone apache è esposto all' esterno per mezzo di un reverse proxy HTTP che inoltra le richieste verso il container, in caso di presenza di agenti remoti, viene esposto anche redis per mezzo di un port forwarding in modo da consentire la comunicazione con macchine esterne

\begin{figure}[H]
    \centering
    \includegraphics[width=0.5\linewidth]{build/container_network.png}
    \caption{networking nel container}
    \label{fig:enter-label}
\end{figure}

% gestione delle integrazioni
Una grande sfida per il progetto e stato determinare una politica con cui gestire le integrazioni, non era possibile ignorarle in fase di progettazione in quanto avrebbe comportato un grave disservizio per il cliente e la re-ingegnerizzazione di ognuna di queste avrebbe comportato un aumento non accettabile dei tempi di sviluppo del progetto, il team ha dunque optato per una riorganizzazione delle integrazioni sistemistiche più frequenti in una raccolta pre installata nel container, questa include:

\begin{itemize}
  \item{discovery automatica dei nodi da aggiungere a monitoraggio}
  \item{backup dei dati applicativi}
  \item{sistema di email spia per verificare il corretto funzionamento della posta}
  \item{tools per il sincronismo di installazioni master-slave}
  \item{monitoraggio di infrastrutture checkmk}
  \item{monitoraggio di infrastrutture VMware}
  \item{monitoraggio del servizio radius }
  \item{monitoraggio del servizio ldap }
  \item{monitoraggio del servizio samba }
\end{itemize}

Non tutte le integrazioni sono presenti all'interno del container di default di conseguenza il container e predisposto per preservare le integrazioni esistenti che vengono importate nel container per mezzo di un bind mount

% gestione delle configurazioni
Le installazioni in produzione prevedono un set di configurazioni custom degli applicativi a bordo come per esempio la configurazione di sanet stesso, del demone di posta, di rancid oppure di postgres, l'architettura a container doveva tener conto di queste configurazioni e riportarle a seguito di una migrazione, per soddisfare il requisito si è deciso di esportare le configurazioni per mezzo di repository \verb|git|

\begin{figure}[H]
    \centering
    \includegraphics[width=1\linewidth]{build/customer_repository.png}
    \caption{gestione delle configurazioni per mezzo di git}
    \label{fig:enter-label}
\end{figure}

Viene predisposta una repository per cliente per garantire la divisione di ambito e all' interno viene replicata la struttura \verb|FHS| con all'interno le configurazioni necessarie ai servizi presenti nel container per funzionare correttamente, Per le prime installazioni e inoltre prevista una repository di configurazioni di default che consenta ai servizi di partire da cui viene creata la versione specifica per il cliente in questione

% perché il container bulky
Per evitare di modificare quanto sviluppato dall'area NAD il container in questione e pensato come architettura all-in-one che pacchettizza tutti i demoni necessari al sistema, in questo modo lo sviluppo di sanet può essere organizzato in maniera indipendente a quello degli artefatti

% perché non docker
Sono state valutate anche soluzioni che prevedessero l'uso della tecnologia docker\cite{docker}, queste tuttavia limitavano molto il workflow dei team sistemistici che non avrebbero potuto modificare agilmente il container in caso di necessità.

\newpage
\subsection{Progettazione del provision: Ansible}

Progettare l'artefatto non e sufficiente, e necessario prevedere le procedure che portino il container in produzione, ne effettuino l'aggiornamento e gestiscano il monitoraggio delle istanze stesse, per adempiere all'obbiettivo il team ha previsto un'architettura centralizzata in cui una macchina provisioner si occupa di gestire lo stato dell'installato, questa viene amministrata dal team stesso.

\begin{figure}[H]
    \centering
    \includegraphics[width=1\linewidth]{build/provisioner_architecture.png}
    \caption{architettura del provisioner}
    \label{fig:enter-label}
\end{figure}

Il provisioner e si occupa dei seguenti compiti:

\begin{itemize}
  \item{creare l'artefatto installabile}
  \item{installare istanza di sanet su infrastruttura on premise }
  \item{aggiornare le istanze di sanet}
  \item{aggiungere al monitoraggio centralizzato un istanza di sanet}
  \item{effettuare operazioni massive come riconfigurazione di componenti dell'infrastruttura applicativa}
\end{itemize}

Il team ha deciso di optare per ansible\cite{ansible} come tecnologia di riferimento per effettuare il provisioning, le procedure vengono implementate come ansible playbooks e il nodo provisioner viene implementato come ansible tower

% progettazione della procedura di build
L'artefatto viene ricreato per mezzo di un playbook che effettua la build del container partendo da uno scheletro base minimale e installando le componenti di necessarie, la build viene controllata dal provisioner e effettuata su una macchina di staging, consentendo al team di effettuare test manuali con l'obbiettivo di collaudare il funzionamento delle sue componenti e integrazioni aggiuntive, nonché testare le nuove features introdotte da un aggiornamento

\begin{figure}[H]
    \centering
    \includegraphics[width=1\linewidth]{build/build_procedure.png}
    \caption{procedura di build}
    \label{fig:enter-label}
\end{figure}

La procedura di build consente di ottenere in maniera programmatica un infrastruttura identica per ogni installazione, e rende possibile la progettazione di operazioni massive in quanto si possono fare assunzioni sull'ambiente runtime dove si va ad operare, questo risulta essere un requisito fondamentale per rendere la manutenzione delle installazioni scalabile

% progettazione della procedura di installazione
Gli artefatti prodotti dalla procedura di build devono poter essere installati in maniera programmatica, è stata dunque prevista una procedura di di installazione di una nuova istanza di sanet a partire dagli artefatti prodotti dalla procedura di build, l'installazione prevede 3 fasi principali:

\begin{itemize}
  \item{preparazione della macchina host e installazione delle componenti principali}
  \item{ottenimento dell'artefatto}
  \item{configurazione dell'artefatto con le configurazioni del cliente specifico}
\end{itemize}

In questa fase viene effettuata inoltre l'inizializzazione del database postgres, necessaria in quanto in fase di installazione la directory dati di postgres viene montata all'esterno del container provocandone la cancellazione dei contenuti, vengono anche inizializzati

% implementazione con ansible
Il playbook che implementa la procedura di installazione e strutturato secondo il pattern strategy, il playbook specifica le macchine in cui eseguire la procedura e richiama un role ansible comportandosi da proxy, è il role stesso che implementa le logiche operative di installazione dello specifico contianer.

\begin{figure}[H]
    \centering
    \includegraphics[width=1\linewidth]{build/proxy_playbook.png}
    \caption{procedura di build}
    \label{fig:enter-label}
\end{figure}

In questo modo il playbook di installazione può essere utilizzato anche in fase di build per effettuare il deploy del container skel

\begin{figure}[H]
    \centering
    \includegraphics[width=0.7\linewidth]{build/build_deploy.png}
    \caption{relazione tra build e installazione}
    \label{fig:enter-label}
\end{figure}

% progettazione della procedura di migrazione
Come detto in precedenza lo stato dell'installato non poteva essere ignorato in questa fase, era fondamentale prevedere una procedura di migrazione che dato uno stato dell'installato potenzialmente ignoto fino alla fase di attuazione della procedura stessa, il team ha adottato un approccio conservativo conscio dei possibili problemi tra cui

\begin{itemize}
  \item{clashing di indirizzamenti ip con quelli del container}
  \item{clashing di nomi all'interno del filesystem}
  \item{stato dei dischi}
  \item{configurazioni discordanti tra il container e quanto presente in produzione}
\end{itemize}

% operazioni pre migrazione
In particolare in questa fase era necessario reperire le configurazioni custom della istanza in questione per poter riconfigurare l'artefatto successivamente, e stato dunque previsto un playbook che si occupasse di creare la repository git e esportare le configurazioni interessanti della istanza,vengono inoltre apportate delle correzioni alle configurazioni per quei software che a seguito di aggiornamenti hanno deprecato dei parametri di configurazione, in caso non sia presente una specifica configurazione viene fornito il default previsto dal team per quel particolare pezzo di software.

Per accelerare la fase di migrate e ridurre il conseguente downtime viene effettuata una pulizia  dei log salvati nel database in modo da ridurre la quantità di dati da esportare.

La procedura di migrate completa si compone delle seguenti fasi


\begin{itemize}
  \item{esportazione delle configurazioni per mezzo di git}
  \item{pulizia dei log a database della istanza in produzione}
  \item{creazione di una istanza di sanet temporanea a bordo della infrastruttura del cliente}
  \item{arresto della vecchia istanza di sanet}
  \item{dump del database}
  \item{restore del database nella nuova infrastruttura a container}
  \item{riconfigurazione della nuova istanza per mezzo della repository git}
\end{itemize}


\begin{figure}[H]
    \centering
    \includegraphics[width=1\linewidth]{build/migration.png}
    \caption{procedura di migrazione}
    \label{fig:enter-label}
\end{figure}

% progettazione della procedura di update
% progettazione del provision

% gestione dei segreti
%Uno dei requisiti per cui e stata scelta la tecnologia a container e rendere il team NOC capace di gestire la procedura di aggiornamento massivo delle installazioni di sanet in maniera centralizzata, il container infatti consente di uniformare la procedura di installazione e aggiornamento in maniera trasparente rispetto alle decisioni prese in area NAD.
% monitoraggio delle istanze dal centro

\chapter{Implementazione}
%Implementazioni delle soluzioni progettate, descrizione delle problematiche incontrate in fase di migrazione e correzioni adottate per sopperire ai problemi

\chapter{sviluppi futuri}
%Possibili sviluppi futuri sia in ambito architetturale sia per quanto riguarda il sistema di mantenimento dell'installato

Il progetto si trova ora nella sua fase finale, restano tuttavia molteplici possibilità per estendere quanto fatto con l'obbiettivo di ridurre il carico di lavoro richiesto ai rispettivi team dai vari workflow e semplificare il software management degli applicativi dei laboratori Marconi.

\begin{itemize}
  \item{estendere la gestione del container agli altri applicativi}
  \item{rivalutare docker con l'obbiettivo di integrarlo nei processi di sviluppo}
  \item{introdurre il testing degli applicativi}
  \item{automatizzare le procedure di build e update}
  \item{sfruttare l'intelligenza artificiale per evitare la software regression per mezzo del testing}
\end{itemize}

% estensione della soluzione agli altri applicativi labs
Quanto fatto per sanet è stato pensato per poter essere esteso agli altri applicativi labs, all'attivo l'area sviluppo mantiene i seguenti software:

\begin{itemize}
  \item{sanet: monitoraggio e allarmistica}
  \item{loggit: accentratore di log}
  \item{logan: analisi di log e allarmistica}
  \item{DHCP-manager: gestione del servizio DHCP}
  \item{datareport: sistema data-lake per l'accentramento dei dati raccolti dai vari applicativi labs}
\end{itemize}

Queste applicazioni mostrano architetture simili o semplificate rispetto a quella di sanet e soffrono delle stesse problematiche, pertanto le procedure e ingegnerizzazioni viste sopra possono essere estese anche a questi software per omogeneizzare la messa in produzione e fornire una interfaccia comune fra sviluppo e produzione.

\subsection{Docker come soluzione per la produzione dell'artefatto}
% docker come soluzione per container
Come accennato in precedenza docker\cite{docker} è stato preso in considerazione in una fase iniziale dello sviluppo ma subito scartato in quanto risultava di primaria importanza mantenere la capacità operativa dei team sistemistici all'interno delle infrastrutture in produzione, tuttavia questo limita fortemente le procedure centralizzate in quanto le operazioni apportate a singole installazioni di sanet fanno si che lo stato dell'installato risulti ignoto, inoltre non è possibile riportare una feature sviluppata per mezzo di un integrazione di sanet alle altre istanze ne effettuare unit o integration testing.

Docker d'altro canto consentirebbe di ridurre la dimensione della codebase delle procedure di deployment e update, evitando al team devops la gestione di esigenze specifiche delle singole applicazioni, per esempio nel caso di sanet successivamente a un aggiornamento può risultare necessario effettuare delle migrazioni sul database \verb|SQL|, queste attualmente vengono gestite dalla procedura di update, tuttavia questo rende la procedura di update specifica per il software sanet e costringe allo sviluppo di una procedura diversa per ogni software, aumentando gli oneri del team devops in fase di manutenzione e aggiornamento della codebase.

Un altro vantaggio di docker\cite{docker} sulla tecnologia LXC\cite{lxc} e quello di fornire all'area NAD una architettura identica a quella di produzione e velocizzare operazioni di integration testing sui sistemi stessi.

Per mostrare le potenzialità offerte da docker è stato creata una demo di un container bulky dell'agente di sanet, in questa versione il container e spoglio delle integrazioni sistemistiche

\begin{lstlisting}[]
# Base image
FROM python:3.11-bookworm

# Variabili di ambiente
ENV REMOTE_INSTALL_DIR=/usr/local/sanet-remote-poller
ENV SANET_REMOTE_VIRTUAL_ENV=/usr/local/sanet-env

RUN python -m venv /usr/local/sanet-env

# Dependency installation
RUN apt-get update && apt-get install -y \
    python3 python3-dev subversion git gcc libsnmp-dev redis-server\
    libcurl4-gnutls-dev libgnutls28-dev unzip snmp snmpd \
    libffi-dev libssl-dev libldap2-dev libsasl2-dev \
    && apt-get clean

WORKDIR /usr/local/src
RUN pip install --upgrade pip setuptools wheel

# sanet component installation
ADD ./sources/sanet-common ./sanet-common
WORKDIR /usr/local/src/sanet-common
RUN pip install -r requirements.txt && \
    pip install .
WORKDIR /usr/local/src

# other components .....

RUN cp /usr/local/src/sanet-remote-poller/etc/global.ini.dist /usr/local/src/sanet-remote-poller/etc/global.ini
RUN cp /usr/local/src/sanet-remote-poller/etc/agents/agent.dist /usr/local/src/sanet-remote-poller/etc/agents/agent

COPY entrypoint.sh .
CMD ["bash","./entrypoint.sh"]
\end{lstlisting}

Docker consente inoltre di differenziare in maniera dinamica le build del container, consentendo una gestione semplificata delle personalizzazioni specifiche per un singolo cliente generandole da un immagine base di sanet:

\begin{figure}[H]
    \centering
    \includegraphics[width=0.5\linewidth]{build/docker_custom_images.png}
    \caption{gestione di immagini custom di sanet sfruttando docker}
    \label{fig:docker_custom_images}
\end{figure}


% automazione della procedura di build
Un forte limite delle procedure sviluppate risiede nel fatto che esse richiedono intervento umano, dopo una modifica il team NAD notifica il team devops della necessita di aggiornamento per mezzo del sistema di ticketing e il team devops prende in carico la richiesta, rigenera gli artefatti e aggiorna l'infrastruttura in produzione.
Questa procedura può essere automatizzata per mezzo di software per l'esecuzione di build distribuite come jenkins\cite{jenkins} e semaphore, questi sistemi si basano su una modellazione della procedura di build e offrono procedure dichiarative per descrivere lo stato dell'artefatto che si vuole ottenere per mezzo della procedura di build stessa.

% testing su istanze distribuite
Questi sistemi sono anche in grado di effettuare integration testing dell'artefatto prodotto in autonomia

\begin{figure}[H]
    \centering
    \includegraphics[width=0.8\linewidth]{build/jenkins_making_sanet.png}
    \caption{Processo di creazione di sanet per mezzo di Jenkins}
    \label{fig:jenkins_making_sanet}
\end{figure}

% generazione di testing scenarios sfruttando IA per evitare casi di regression
L'utilizzo di infrastruttura per il build e il testing di sanet apre la strada a scenari dove e possibile utilizzare sistemi di intelligenza artificiale per la generazione di test scenario, l'obbiettivo di tale strategia e quello di prevenire la problematica detta software regression, dove un bug precedentemente scovato e risolto si ripresenta a seguito di un aggiornamento successivo.

Il sistema sfrutterebbe la base documentale di ticket aperti all'area NAD nel corso degli anni per generare test cases nella forma di jenkins pipelines, questi verrebbero poi validati dal team devops e inseriti nella procedura di testing post build di sanet.

\chapter{Conclusioni}
%Considerazioni sulle performance, obbiettivi conseguiti, sum up generale

La gestione del software life-cycle è un tema che un azienda non può trascurare nello sviluppo software industriale dato che le problematiche generate da una cattiva gestione del software possono portare a situazioni di scarsa sicurezza, perdita di conoscenza dello stato della produzione e rallentamenti continui nelle fasi di sviluppo e update del software, uno dei motivi per cui questo aspetto dello sviluppo viene sempre trascurato risiede negli alti costi di gestione in termini di ore uomo delle procedure di build,test e deploy che per quanto fondamentali vengono viste come una esigenza marginale su cui non investire risorse.

La chiave per affrontare questo problema risulta essere l'automazione delle suddette procedure, il team non deve investire tempo nella creazione di artefatti o nel aggiornare la produzione, ma nel progettare come automatizzare queste procedure seguendo una politica d.r.y. (don't repeat yourself)

Con l'introduzione dell'automazione in  questi processi, soprattutto per quanto riguarda l'aggiornamento della produzione acquisisce ancora più importanza la progettazione del testing, non solo per quanto riguarda la correttezza della codebase ma anche per il corretto funzionamento delle integrazioni a runtime, simulare ambienti distribuiti e stress test di carico diventa un problema altrettanto imponente in quanto non si e più fautori della messa in produzione ma la si delega a sistemi automatici.

In questo contesto la conoscenza dello stato delle istanze in produzione attraverso il monitoraggio e la rimozione di eventuali workflow che modifichino lo stato della produzione a runtime risulta fondamentale per garantire che il testing fornisca risultati chiari e certi.





\backmatter
\addcontentsline{toc}{chapter}{Bibliografia}
\bibliographystyle{unsrt}
% i riferimenti bibliografici, che devono essere almeno 20 per una tesi triennale ed almeno 30 per una della magistrale, si scaricano da qui https://dblp.uni-trier.de/search/
% e si aggiungono al file bibliografia.bib, dopodichè si citano opportuanamente nel testo della tesi con \cite{label}  dove label è il primo elemento di ogni rif. bibliografico subito dopo la parentesi graffa aperta, e.g. DBLP:books/daglib/0087929 (vedi file .bib sopra menzionato)
%\bibliography{bibliography}

\end{document}
