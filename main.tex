\documentclass[12pt,a4paper,twoside,openright]{book}

% instead of paragraph indentation on first line and no spacing
% makes no indentation and spacing
\usepackage{parskip}

\usepackage[utf8]{inputenc}
\usepackage[english]{babel}
\usepackage[T1]{fontenc}

\usepackage{style/isi_style_lt}

\usepackage{amsmath,amsfonts,amssymb,amsthm}
\usepackage{caption}
\usepackage[usenames]{color}
\usepackage{enumerate}
\usepackage{fancyhdr}
\usepackage{fancyvrb}
\usepackage{float}
\usepackage{booktabs}
\usepackage{indentfirst}
\usepackage{listings}
\usepackage{marvosym}
\usepackage{multicol}
\usepackage{sectsty}
\usepackage{tocloft}
\usepackage{microtype}
\usepackage[table]{xcolor}
\usepackage{url}
\usepackage{hyperref}
\usepackage[toc]{appendix}

\usepackage{dsfont}
\usepackage{comment}
\usepackage{multirow}
\usepackage{hhline}
\usepackage{adjustbox}
\usepackage{tkz-tab}
\usepackage{tkz-kiviat,numprint}
\usepackage{pgfplotstable}
\usepackage{pgfplots}
\usepackage{subcaption}

\usepackage{tablefootnote}

\usepackage{graphicx}
\usepackage{tikz}
\usepackage{forest}
\usetikzlibrary{trees,positioning,shapes,shadows,arrows.meta}

\definecolor{bar1}{HTML}{FFEF9F}
\definecolor{bar2}{HTML}{FFCFD2}
\definecolor{bar3}{HTML}{CFBAF0}
\definecolor{bar4}{HTML}{90DBF4}
\definecolor{line1}{HTML}{BC4749}
\definecolor{line2}{HTML}{168AAD}
\definecolor{line3}{HTML}{7B2CBF}

\definecolor{radar1}{HTML}{E3B505}
\definecolor{radar2}{HTML}{34B2E4}
\definecolor{radar3}{HTML}{065381}
\definecolor{radar4}{HTML}{59A96A}
\definecolor{radar5}{HTML}{E34856}
\definecolor{radar6}{HTML}{FE912A}

\newcommand*{\paramsbox}[2]{{\small \colorbox{white!#1!red}{\color{black} #2}}}
\newcommand*{\carbonbox}[2]{{\small \colorbox{white!#1!green}{\color{black} #2}}}



    \usepackage{adjustbox} % Used to constrain images to a maximum size 
    \usepackage{textcomp} % defines textquotesingle
    % Hack from http://tex.stackexchange.com/a/47451/13684:
    \AtBeginDocument{%
        \def\PYZsq{\textquotesingle}% Upright quotes in Pygmentized code
    }
    \usepackage{upquote} % Upright quotes for verbatim code
    \usepackage{eurosym} % defines \euro
    \usepackage[mathletters]{ucs} % Extended unicode (utf-8) support
    \usepackage{grffile} % extends the file name processing of package graphics 
                         % to support a larger range 
    \usepackage{longtable} % longtable support required by pandoc >1.10
    \usepackage{booktabs}  % table support for pandoc > 1.12.2
    \usepackage[inline]{enumitem} % IRkernel/repr support (it uses the enumerate* environment)
    \usepackage[normalem]{ulem} % ulem is needed to support strikethroughs (\sout)

    
    % Colors for the hyperref package
    \definecolor{urlcolor}{rgb}{0,.145,.698}
    \definecolor{linkcolor}{rgb}{.71,0.21,0.01}
    \definecolor{citecolor}{rgb}{.12,.54,.11}

    % ANSI colors
    \definecolor{ansi-black}{HTML}{3E424D}
    \definecolor{ansi-black-intense}{HTML}{282C36}
    \definecolor{ansi-red}{HTML}{E75C58}
    \definecolor{ansi-red-intense}{HTML}{B22B31}
    \definecolor{ansi-green}{HTML}{00A250}
    \definecolor{ansi-green-intense}{HTML}{007427}
    \definecolor{ansi-yellow}{HTML}{DDB62B}
    \definecolor{ansi-yellow-intense}{HTML}{B27D12}
    \definecolor{ansi-blue}{HTML}{208FFB}
    \definecolor{ansi-blue-intense}{HTML}{0065CA}
    \definecolor{ansi-magenta}{HTML}{D160C4}
    \definecolor{ansi-magenta-intense}{HTML}{A03196}
    \definecolor{ansi-cyan}{HTML}{60C6C8}
    \definecolor{ansi-cyan-intense}{HTML}{258F8F}
    \definecolor{ansi-white}{HTML}{C5C1B4}
    \definecolor{ansi-white-intense}{HTML}{A1A6B2}

    % commands and environments needed by pandoc snippets
    % extracted from the output of `pandoc -s`
    \providecommand{\tightlist}{%
      \setlength{\itemsep}{0pt}\setlength{\parskip}{0pt}}
    \DefineVerbatimEnvironment{Highlighting}{Verbatim}{commandchars=\\\{\}}
    % Add ',fontsize=\small' for more characters per line
    \newenvironment{Shaded}{}{}
    \newcommand{\KeywordTok}[1]{\textcolor[rgb]{0.00,0.44,0.13}{\textbf{{#1}}}}
    \newcommand{\DataTypeTok}[1]{\textcolor[rgb]{0.56,0.13,0.00}{{#1}}}
    \newcommand{\DecValTok}[1]{\textcolor[rgb]{0.25,0.63,0.44}{{#1}}}
    \newcommand{\BaseNTok}[1]{\textcolor[rgb]{0.25,0.63,0.44}{{#1}}}
    \newcommand{\FloatTok}[1]{\textcolor[rgb]{0.25,0.63,0.44}{{#1}}}
    \newcommand{\CharTok}[1]{\textcolor[rgb]{0.25,0.44,0.63}{{#1}}}
    \newcommand{\StringTok}[1]{\textcolor[rgb]{0.25,0.44,0.63}{{#1}}}
    \newcommand{\CommentTok}[1]{\textcolor[rgb]{0.38,0.63,0.69}{\textit{{#1}}}}
    \newcommand{\OtherTok}[1]{\textcolor[rgb]{0.00,0.44,0.13}{{#1}}}
    \newcommand{\AlertTok}[1]{\textcolor[rgb]{1.00,0.00,0.00}{\textbf{{#1}}}}
    \newcommand{\FunctionTok}[1]{\textcolor[rgb]{0.02,0.16,0.49}{{#1}}}
    \newcommand{\RegionMarkerTok}[1]{{#1}}
    \newcommand{\ErrorTok}[1]{\textcolor[rgb]{1.00,0.00,0.00}{\textbf{{#1}}}}
    \newcommand{\NormalTok}[1]{{#1}}
    
    % Additional commands for more recent versions of Pandoc
    \newcommand{\ConstantTok}[1]{\textcolor[rgb]{0.53,0.00,0.00}{{#1}}}
    \newcommand{\SpecialCharTok}[1]{\textcolor[rgb]{0.25,0.44,0.63}{{#1}}}
    \newcommand{\VerbatimStringTok}[1]{\textcolor[rgb]{0.25,0.44,0.63}{{#1}}}
    \newcommand{\SpecialStringTok}[1]{\textcolor[rgb]{0.73,0.40,0.53}{{#1}}}
    \newcommand{\ImportTok}[1]{{#1}}
    \newcommand{\DocumentationTok}[1]{\textcolor[rgb]{0.73,0.13,0.13}{\textit{{#1}}}}
    \newcommand{\AnnotationTok}[1]{\textcolor[rgb]{0.38,0.63,0.69}{\textbf{\textit{{#1}}}}}
    \newcommand{\CommentVarTok}[1]{\textcolor[rgb]{0.38,0.63,0.69}{\textbf{\textit{{#1}}}}}
    \newcommand{\VariableTok}[1]{\textcolor[rgb]{0.10,0.09,0.49}{{#1}}}
    \newcommand{\ControlFlowTok}[1]{\textcolor[rgb]{0.00,0.44,0.13}{\textbf{{#1}}}}
    \newcommand{\OperatorTok}[1]{\textcolor[rgb]{0.40,0.40,0.40}{{#1}}}
    \newcommand{\BuiltInTok}[1]{{#1}}
    \newcommand{\ExtensionTok}[1]{{#1}}
    \newcommand{\PreprocessorTok}[1]{\textcolor[rgb]{0.74,0.48,0.00}{{#1}}}
    \newcommand{\AttributeTok}[1]{\textcolor[rgb]{0.49,0.56,0.16}{{#1}}}
    \newcommand{\InformationTok}[1]{\textcolor[rgb]{0.38,0.63,0.69}{\textbf{\textit{{#1}}}}}
    \newcommand{\WarningTok}[1]{\textcolor[rgb]{0.38,0.63,0.69}{\textbf{\textit{{#1}}}}}
    
    
    % Define a nice break command that doesn't care if a line doesn't already
    % exist.
    \def\br{\hspace*{\fill} \\* }
    % Math Jax compatability definitions
    \def\gt{>}
    \def\lt{<}
    % Document parameters
    \title{Untitled1}
    
    
    

    % Pygments definitions
    
\makeatletter
\def\PY@reset{\let\PY@it=\relax \let\PY@bf=\relax%
    \let\PY@ul=\relax \let\PY@tc=\relax%
    \let\PY@bc=\relax \let\PY@ff=\relax}
\def\PY@tok#1{\csname PY@tok@#1\endcsname}
\def\PY@toks#1+{\ifx\relax#1\empty\else%
    \PY@tok{#1}\expandafter\PY@toks\fi}
\def\PY@do#1{\PY@bc{\PY@tc{\PY@ul{%
    \PY@it{\PY@bf{\PY@ff{#1}}}}}}}
\def\PY#1#2{\PY@reset\PY@toks#1+\relax+\PY@do{#2}}

\expandafter\def\csname PY@tok@w\endcsname{\def\PY@tc##1{\textcolor[rgb]{0.73,0.73,0.73}{##1}}}
\expandafter\def\csname PY@tok@c\endcsname{\let\PY@it=\textit\def\PY@tc##1{\textcolor[rgb]{0.25,0.50,0.50}{##1}}}
\expandafter\def\csname PY@tok@cp\endcsname{\def\PY@tc##1{\textcolor[rgb]{0.74,0.48,0.00}{##1}}}
\expandafter\def\csname PY@tok@k\endcsname{\let\PY@bf=\textbf\def\PY@tc##1{\textcolor[rgb]{0.00,0.50,0.00}{##1}}}
\expandafter\def\csname PY@tok@kp\endcsname{\def\PY@tc##1{\textcolor[rgb]{0.00,0.50,0.00}{##1}}}
\expandafter\def\csname PY@tok@kt\endcsname{\def\PY@tc##1{\textcolor[rgb]{0.69,0.00,0.25}{##1}}}
\expandafter\def\csname PY@tok@o\endcsname{\def\PY@tc##1{\textcolor[rgb]{0.40,0.40,0.40}{##1}}}
\expandafter\def\csname PY@tok@ow\endcsname{\let\PY@bf=\textbf\def\PY@tc##1{\textcolor[rgb]{0.67,0.13,1.00}{##1}}}
\expandafter\def\csname PY@tok@nb\endcsname{\def\PY@tc##1{\textcolor[rgb]{0.00,0.50,0.00}{##1}}}
\expandafter\def\csname PY@tok@nf\endcsname{\def\PY@tc##1{\textcolor[rgb]{0.00,0.00,1.00}{##1}}}
\expandafter\def\csname PY@tok@nc\endcsname{\let\PY@bf=\textbf\def\PY@tc##1{\textcolor[rgb]{0.00,0.00,1.00}{##1}}}
\expandafter\def\csname PY@tok@nn\endcsname{\let\PY@bf=\textbf\def\PY@tc##1{\textcolor[rgb]{0.00,0.00,1.00}{##1}}}
\expandafter\def\csname PY@tok@ne\endcsname{\let\PY@bf=\textbf\def\PY@tc##1{\textcolor[rgb]{0.82,0.25,0.23}{##1}}}
\expandafter\def\csname PY@tok@nv\endcsname{\def\PY@tc##1{\textcolor[rgb]{0.10,0.09,0.49}{##1}}}
\expandafter\def\csname PY@tok@no\endcsname{\def\PY@tc##1{\textcolor[rgb]{0.53,0.00,0.00}{##1}}}
\expandafter\def\csname PY@tok@nl\endcsname{\def\PY@tc##1{\textcolor[rgb]{0.63,0.63,0.00}{##1}}}
\expandafter\def\csname PY@tok@ni\endcsname{\let\PY@bf=\textbf\def\PY@tc##1{\textcolor[rgb]{0.60,0.60,0.60}{##1}}}
\expandafter\def\csname PY@tok@na\endcsname{\def\PY@tc##1{\textcolor[rgb]{0.49,0.56,0.16}{##1}}}
\expandafter\def\csname PY@tok@nt\endcsname{\let\PY@bf=\textbf\def\PY@tc##1{\textcolor[rgb]{0.00,0.50,0.00}{##1}}}
\expandafter\def\csname PY@tok@nd\endcsname{\def\PY@tc##1{\textcolor[rgb]{0.67,0.13,1.00}{##1}}}
\expandafter\def\csname PY@tok@s\endcsname{\def\PY@tc##1{\textcolor[rgb]{0.73,0.13,0.13}{##1}}}
\expandafter\def\csname PY@tok@sd\endcsname{\let\PY@it=\textit\def\PY@tc##1{\textcolor[rgb]{0.73,0.13,0.13}{##1}}}
\expandafter\def\csname PY@tok@si\endcsname{\let\PY@bf=\textbf\def\PY@tc##1{\textcolor[rgb]{0.73,0.40,0.53}{##1}}}
\expandafter\def\csname PY@tok@se\endcsname{\let\PY@bf=\textbf\def\PY@tc##1{\textcolor[rgb]{0.73,0.40,0.13}{##1}}}
\expandafter\def\csname PY@tok@sr\endcsname{\def\PY@tc##1{\textcolor[rgb]{0.73,0.40,0.53}{##1}}}
\expandafter\def\csname PY@tok@ss\endcsname{\def\PY@tc##1{\textcolor[rgb]{0.10,0.09,0.49}{##1}}}
\expandafter\def\csname PY@tok@sx\endcsname{\def\PY@tc##1{\textcolor[rgb]{0.00,0.50,0.00}{##1}}}
\expandafter\def\csname PY@tok@m\endcsname{\def\PY@tc##1{\textcolor[rgb]{0.40,0.40,0.40}{##1}}}
\expandafter\def\csname PY@tok@gh\endcsname{\let\PY@bf=\textbf\def\PY@tc##1{\textcolor[rgb]{0.00,0.00,0.50}{##1}}}
\expandafter\def\csname PY@tok@gu\endcsname{\let\PY@bf=\textbf\def\PY@tc##1{\textcolor[rgb]{0.50,0.00,0.50}{##1}}}
\expandafter\def\csname PY@tok@gd\endcsname{\def\PY@tc##1{\textcolor[rgb]{0.63,0.00,0.00}{##1}}}
\expandafter\def\csname PY@tok@gi\endcsname{\def\PY@tc##1{\textcolor[rgb]{0.00,0.63,0.00}{##1}}}
\expandafter\def\csname PY@tok@gr\endcsname{\def\PY@tc##1{\textcolor[rgb]{1.00,0.00,0.00}{##1}}}
\expandafter\def\csname PY@tok@ge\endcsname{\let\PY@it=\textit}
\expandafter\def\csname PY@tok@gs\endcsname{\let\PY@bf=\textbf}
\expandafter\def\csname PY@tok@gp\endcsname{\let\PY@bf=\textbf\def\PY@tc##1{\textcolor[rgb]{0.00,0.00,0.50}{##1}}}
\expandafter\def\csname PY@tok@go\endcsname{\def\PY@tc##1{\textcolor[rgb]{0.53,0.53,0.53}{##1}}}
\expandafter\def\csname PY@tok@gt\endcsname{\def\PY@tc##1{\textcolor[rgb]{0.00,0.27,0.87}{##1}}}
\expandafter\def\csname PY@tok@err\endcsname{\def\PY@bc##1{\setlength{\fboxsep}{0pt}\fcolorbox[rgb]{1.00,0.00,0.00}{1,1,1}{\strut ##1}}}
\expandafter\def\csname PY@tok@kc\endcsname{\let\PY@bf=\textbf\def\PY@tc##1{\textcolor[rgb]{0.00,0.50,0.00}{##1}}}
\expandafter\def\csname PY@tok@kd\endcsname{\let\PY@bf=\textbf\def\PY@tc##1{\textcolor[rgb]{0.00,0.50,0.00}{##1}}}
\expandafter\def\csname PY@tok@kn\endcsname{\let\PY@bf=\textbf\def\PY@tc##1{\textcolor[rgb]{0.00,0.50,0.00}{##1}}}
\expandafter\def\csname PY@tok@kr\endcsname{\let\PY@bf=\textbf\def\PY@tc##1{\textcolor[rgb]{0.00,0.50,0.00}{##1}}}
\expandafter\def\csname PY@tok@bp\endcsname{\def\PY@tc##1{\textcolor[rgb]{0.00,0.50,0.00}{##1}}}
\expandafter\def\csname PY@tok@fm\endcsname{\def\PY@tc##1{\textcolor[rgb]{0.00,0.00,1.00}{##1}}}
\expandafter\def\csname PY@tok@vc\endcsname{\def\PY@tc##1{\textcolor[rgb]{0.10,0.09,0.49}{##1}}}
\expandafter\def\csname PY@tok@vg\endcsname{\def\PY@tc##1{\textcolor[rgb]{0.10,0.09,0.49}{##1}}}
\expandafter\def\csname PY@tok@vi\endcsname{\def\PY@tc##1{\textcolor[rgb]{0.10,0.09,0.49}{##1}}}
\expandafter\def\csname PY@tok@vm\endcsname{\def\PY@tc##1{\textcolor[rgb]{0.10,0.09,0.49}{##1}}}
\expandafter\def\csname PY@tok@sa\endcsname{\def\PY@tc##1{\textcolor[rgb]{0.73,0.13,0.13}{##1}}}
\expandafter\def\csname PY@tok@sb\endcsname{\def\PY@tc##1{\textcolor[rgb]{0.73,0.13,0.13}{##1}}}
\expandafter\def\csname PY@tok@sc\endcsname{\def\PY@tc##1{\textcolor[rgb]{0.73,0.13,0.13}{##1}}}
\expandafter\def\csname PY@tok@dl\endcsname{\def\PY@tc##1{\textcolor[rgb]{0.73,0.13,0.13}{##1}}}
\expandafter\def\csname PY@tok@s2\endcsname{\def\PY@tc##1{\textcolor[rgb]{0.73,0.13,0.13}{##1}}}
\expandafter\def\csname PY@tok@sh\endcsname{\def\PY@tc##1{\textcolor[rgb]{0.73,0.13,0.13}{##1}}}
\expandafter\def\csname PY@tok@s1\endcsname{\def\PY@tc##1{\textcolor[rgb]{0.73,0.13,0.13}{##1}}}
\expandafter\def\csname PY@tok@mb\endcsname{\def\PY@tc##1{\textcolor[rgb]{0.40,0.40,0.40}{##1}}}
\expandafter\def\csname PY@tok@mf\endcsname{\def\PY@tc##1{\textcolor[rgb]{0.40,0.40,0.40}{##1}}}
\expandafter\def\csname PY@tok@mh\endcsname{\def\PY@tc##1{\textcolor[rgb]{0.40,0.40,0.40}{##1}}}
\expandafter\def\csname PY@tok@mi\endcsname{\def\PY@tc##1{\textcolor[rgb]{0.40,0.40,0.40}{##1}}}
\expandafter\def\csname PY@tok@il\endcsname{\def\PY@tc##1{\textcolor[rgb]{0.40,0.40,0.40}{##1}}}
\expandafter\def\csname PY@tok@mo\endcsname{\def\PY@tc##1{\textcolor[rgb]{0.40,0.40,0.40}{##1}}}
\expandafter\def\csname PY@tok@ch\endcsname{\let\PY@it=\textit\def\PY@tc##1{\textcolor[rgb]{0.25,0.50,0.50}{##1}}}
\expandafter\def\csname PY@tok@cm\endcsname{\let\PY@it=\textit\def\PY@tc##1{\textcolor[rgb]{0.25,0.50,0.50}{##1}}}
\expandafter\def\csname PY@tok@cpf\endcsname{\let\PY@it=\textit\def\PY@tc##1{\textcolor[rgb]{0.25,0.50,0.50}{##1}}}
\expandafter\def\csname PY@tok@c1\endcsname{\let\PY@it=\textit\def\PY@tc##1{\textcolor[rgb]{0.25,0.50,0.50}{##1}}}
\expandafter\def\csname PY@tok@cs\endcsname{\let\PY@it=\textit\def\PY@tc##1{\textcolor[rgb]{0.25,0.50,0.50}{##1}}}

\def\PYZbs{\char`\\}
\def\PYZus{\char`\_}
\def\PYZob{\char`\{}
\def\PYZcb{\char`\}}
\def\PYZca{\char`\^}
\def\PYZam{\char`\&}
\def\PYZlt{\char`\<}
\def\PYZgt{\char`\>}
\def\PYZsh{\char`\#}
\def\PYZpc{\char`\%}
\def\PYZdl{\char`\$}
\def\PYZhy{\char`\-}
\def\PYZsq{\char`\'}
\def\PYZdq{\char`\"}
\def\PYZti{\char`\~}
% for compatibility with earlier versions
\def\PYZat{@}
\def\PYZlb{[}
\def\PYZrb{]}
\makeatother


    % Exact colors from NB
    \definecolor{incolor}{rgb}{0.0, 0.0, 0.5}
    \definecolor{outcolor}{rgb}{0.545, 0.0, 0.0}




\hypersetup{%
	pdfpagemode={UseOutlines},
	bookmarksopen,
	pdfstartview={FitH},
	colorlinks,
	linkcolor={black},
	citecolor={black},
	urlcolor={black}
}

\AtBeginDocument{%
	\renewcommand{\contentsname}{Indice}%Indice
	\renewcommand\tablename{Tabella}%Tabella
	\renewcommand{\listtablename}{Elenco delle tabelle}%Elenco delle tabelle
	\renewcommand\figurename{Figura}%Figura
	\renewcommand{\listfigurename}{Elenco delle figure}%Elenco delle figure
	\renewcommand{\lstlistingname}{Codice}%Listato
	\renewcommand{\chaptername}{Capitolo}%Capitolo
	\renewcommand{\refname}{Riferimenti}%Riferimenti
	\renewcommand{\bibname}{Bibliografia}%Bibliografia
}

\definecolor{dkgreen}{rgb}{0,0.6,0}
\definecolor{gray}{rgb}{0.5,0.5,0.5}
\definecolor{mauve}{rgb}{0.58,0,0.82}

\lstset{
  frame=single,
  captionpos=b,
  language=Java,
  aboveskip=3mm,
  belowskip=3mm,
  showstringspaces=false,
  columns=flexible,
  basicstyle={\small\ttfamily},
  numbers=none,
  numberstyle=\tiny\color{gray},
  keywordstyle=\color{blue},
  commentstyle=\color{dkgreen},
  stringstyle=\color{mauve},
  breaklines=true,
  breakatwhitespace=true,
  tabsize=3
}

\makeatletter
\def\cleardoublepage{
	\clearpage\if@twoside \ifodd\c@page\else
	\hbox{}
	\thispagestyle{empty}
	\newpage
	\if@twocolumn\hbox{}\newpage\fi\fi\fi
}

\makeatother

\setlength{\textwidth}{14cm}
\setlength{\textheight}{21cm}
\setlength{\footskip}{3cm}

\setlength{\hoffset}{0pt}
\setlength{\voffset}{0pt}

\setlength{\oddsidemargin}{1cm}
\setlength{\evensidemargin}{1cm}

\universita{Alma Mater Studiorum -- Università di Bologna}

\scuola{Scuola di Ingegneria}

\corsodilaurea{Corso di Laurea Magistrale in Ingegneria Informatica}

\titolo{Sistemi per la gestione scalabile del software life-cycle in applicativi di monitoraggio}

\materia{laboratorio di amministrazione di sistemi}

\laureando{Matteo Longhi}

\relatore[\footnotesize Chiar.mo Prof. Ing.\normalsize]{Marco Prandini}
%\correlatoreA[\footnotesize Prof.\normalsize]{}
%\correlatoreB[\footnotesize Dott. Ing.\normalsize]{}

\sessione{Seconda} % 'Prima', 'Seconda', 'Terza'

\annoaccademico{2024 -- 2025}

\parolechiave %devono essere 5 keywordsrolechiave,
{Provisioning}
{scalabilità}
{monitoraggio}
{automazione}
{software life-cycle}


\dedica{\emph{ai ragazzi del Network operation center dei laboratori Marconi e ai gorilla che mi hanno supportato in questo percorso}}

\makeindex

\begin{document}

\frontmatter

\maketitle

\chapter*{Sommario}
\markboth{Sommario}{Sommario}

Lo sviluppo software per risultare efficace necessita di una adeguata strategia di manutenzione e monitoraggio delle istanze in produzione per prevenire situazioni di software erosion. Il mio elaborato di tesi punta a mostrare quanto e stato svolto presso i Laboratori Marconi per quanto riguarda il software life-cycle del sistema di monitoraggio aziendale: Sanet.

Nel elaborato verranno affrontate problematiche quali:

\begin{itemize}
  \item Riorganizzazione dell'architettura di produzione del software
  \item Progettazione delle procedure di update del software
  \item Pianificazione e esecuzione delle operazioni di migrazione delle istanze di produzione
\end{itemize}

L'obbiettivo del progetto che ho condotto presso i Laboratori Marconi e' stato allineare le installazioni di Sanet, a una unica architettura di produzione che fosse in grado di coprire tutte le varie esigenze di business in termini di servizi erogati. L'architettura in questione deve essere replicabile in maniera programmatica.

Le sopracitate problematiche verranno esplorate con un occhio dedicato alla scalabilità e malleabilità delle operazioni e al mantenimento dei processi produttivi aziendali, con l'obbiettivo di rendere il Network operation center capace di gestire le operazioni regolari di manutenzione del software

Il progetto ha comportato un ampia fase di analisi e studio dei workflow aziendali tra cui:

\begin{itemize}
  \item Workflow operativi del network operation center, intercettazione e gestione degli allarmi generati da sanet e procedure di escalation
  \item Workflow operativi sistemistici legati alla manutenzione di Sanet e utilizzo come pulizia dei log e aggiunta a monitoraggio degli apparati
  \item Workflow di sviluppo di integrazioni e componenti di sanet sia da parte dei team sistemistici che dell'area di sviluppo
\end{itemize}

Successivamente sono stati presi in analisi lo stato delle produzioni di sanet, identificate le situazioni critiche, i casi limite e le particolarità delle singole istanze, questo per avere coscienza di cosa sia necessario nell'architettura di produzione finale e definire il minimo insieme che contenesse tutte le features richieste per erogare il servizio di monitoraggio.

A seguito la progettazione e sviluppo della nuova architettura e del conseguente processo di migrazione delle istanze in produzione alla nuova architettura. Con la conseguente produzione della documentazione operativa e di sviluppo per la manutenzione e utilizzo del progetto.

\newpage

\tableofcontents

\newpage

\listoffigures

\mainmatter

\pagestyle{fancy}
\fancyhead[LO]{\nouppercase{\rightmark}}
\fancyhead[RE]{\nouppercase{\leftmark}}
\fancyhead[LE,RO]{\thepage}
\fancyfoot{}

\chapter{Introduzione, background teorico e tecnologico}

La manutenzione e il monitoraggio di un software in produzione e fondamentale per prevenire fenomeni di software erosion, gestire il software life-cycle nelle grandi installazioni e' sempre molto costoso, specialmente se il core business non e lo sviluppo software ma il servizio erogato tramite esso.

Il problema si presenta in maniera ancora maggiore quando la separazione dei contesti dovuta all'espansione del parco clienti porta a un crescere sempre maggiore del numero di istanze del software in produzione che erogano il servizio, gli effetti più comuni della software erosion sono:

\begin{itemize}
    \item dipendenze software non aggiornate che portano a incompatibilità applicative con standard dell'industria
    \item degradazione delle performance dovuta al riempimento dei dischi
    \item sistemi insicuri, dovuti al degrado di protocolli di cifratura utilizzati per le comunicazioni
\end{itemize}

Queste problematiche possono diventare causa di forti rallentamenti al workflow aziendale, specialmente se ignorati per molto tempo, impedendo agli utenti amministratori l'utilizzo del software dato che questi si vedono costretti a impiegare le loro risorse in operazioni di manutenzione del software che spesso e volentieri non sono risolutive del problema.

Un esempio può essere l'installazione manuale di dipendenze applicative, rese obsolete da aggiornamenti dell'ambiente di esecuzione ma necessarie all'applicazione per funzionare risultando in ambienti non solo instabili ma anche insicuri.

La software erosion incide anche sullo sviluppo del software in produzione, dato che l'aggiornamento di istanze "dimenticate" può richiedere troppe risorse invogliando cosi all'implementazione di workaround pur di soddisfare esigenze di servizio. In queste situazioni l'istanza in produzione e soggetta a una "morsa a tenaglia" dove il ciclo di rilascio del software si deve scontrare con integrazioni applicative sviluppate dagli utenti sistemisti che sfruttano API "non convenzionali" per arricchire le funzionalità del sistema stesso.

Il problema risulta accentuato se lo sviluppo non prevede la produzione e l'utilizzo di artefatti per la messa in produzione del software, in questo caso le infrastrutture in produzione sono alla merce di procedure di installazione manuali che portano a una frammentazione degli ambienti in produzione. La creazione di artefatti, specialmente per sistemi software complessi, non deve prevedere solo la messa in funzione delle componenti applicative sviluppate a doc ma anche di tutti i possibili servizi di cui l'applicazione necessita come:

\begin{itemize}
    \item servizi di data storage (databases,demoni cache)
    \item servizi web
    \item demoni per l'esecuzione di manutenzioni programmate
\end{itemize}

Per affrontare il problema sono nati i container, ambienti che sfruttano la funzionalità del kernel linux namespaces per isolare processi applicativi su differenti livelli. Questi sistemi offrono spesso approcci dichiarativi al problema consentendo allo sviluppatore la riproducibilità dell'ambiente di esecuzione e testing. Inoltre consentono di uniformare l'interfaccia fra sviluppo e operazioni di devops e rendere lo sviluppo consapevole dell'ambiente di produzione già in fase di progettazione.

I container offrono la possibilità allo sviluppo di progettare il funzionamento di features applicative che necessitano di un determinato supporto runtime dall'ambiente di produzione, per esempio assicurarsi della raggiungibilità via rete dei componenti dell'applicazione, creare ambienti di test in agilità simulando interazioni fra vari nodi in rete, testare integrazioni con altri ambienti prima di aggiornare la produzione

\chapter{Analisi dei requisiti}
%Requisiti di progetto e obbiettivo finale richiesto dalla missione di business

L'attività svolta presso i laboratori Marconi aveva come obbiettivo finale rendere i membri del network operation center team (NOC in breve) capaci di svolgere le operazioni di manutenzione delle installazioni del software di monitoraggio aziendale Sanet, in particolare:

\begin{itemize}
\item svolgere in autonomia operazioni di aggiornamento distribuito di sanet
\item programmare e organizzare attività di installazione di istanze di sanet
\item svolgere monitoraggio delle performance e conseguenti attività di manutenzione come pulizia di log applicativi, bilanciamenti di carico, aggiunta di nodi a monitoraggio
\end{itemize}

Questo con l'obbiettivo di poter successivamente spostare la responsabilità della gestione delle istanze in produzione al team NOC, e evitare che i membri dei singoli team di sistemisti si debbano occupare di tale attività, liberando risorse da dedicare a task sistemistici richiesti dai clienti.

E' stato inoltre richiesto che fossero revisionate le documentazioni tecniche riguardanti le attività di monitoraggio svolte presso i vari clienti, in modo da uniformare la base documentale e renderne più semplice l'accesso in maniera programmatica.

I requisti richiesti hanno come scopo ultimo quello di ridurre la quantità di ore uomo complessive spese in operazioni di manutenzione di Sanet, ridurre l'entropia delle installazioni in produzione accentrandone la gestione a un unico team

\chapter{Analisi del problema di business}
%Ampia descrizione delle problematiche operative (workflow aziendali non più manutenibili) e di sviluppo (mantenimento dell'applicativo, automazione delle operazioni di ciclo di vita del software) overview di ciò che già e stato fatto e come viene gestito il software in produzione

\subsection{Analisi dell'applicativo}
% Analisi della situazione logistica aziendale, workflow di gestione di sanet, capacita operativa del team NOC in tale contesto

Il business principale dei laboratori Marconi consiste nel monitoraggio di apparati e servizi applicativi, per erogare tale servizio e non dipendere da software esterno l'azienda ha deciso di sviluppare la sua soluzione di monitoraggio proprietaria: Sanet.

Dal punto di vista architetturale Sanet e un sistema distribuito basato su Simple network message protocol (\verb|SNMP|), composto da processi demoni di varia natura tra cui:

\begin{itemize}
    \item demoni per lo storage dei dati
    \item demoni per il recupero dei dati stessi
    \item demoni per la gestione delle notifiche
\end{itemize}

\begin{figure}[H]
    \centering
    \includegraphics[width=1\linewidth]{build/sanet_architecture.png}
    \caption{architettura di sanet}
    \label{fig:enter-label}
\end{figure}

% architettura applicativa
L'architettura si compone di un demone principale sanetd che svolge i seguenti compiti:

\begin{itemize}
  \item gestione del flusso dati in ingresso e della ripartizione e scheduling dei task da eseguire per effettuare il monitoraggio
  \item salvataggio di log applicativi e stato dei controlli su storage stabile
  \item gestione delle timelines generate dai task di monitoraggio
  \item gestione del flusso in ingresso per il demone di gestione delle notifiche
\end{itemize}

% datagroups
I task generati dal demone principale vengono denominati \verb|datagroups|, un datagroup e la composizione di tre elementi principali:

\begin{itemize}
  \item \verb|datasource| sorgente dei dati, specifica le modalità con cui i dati devono essere reperiti, per esempio possono essere specificate OID SNMP, files, risorse web
  \item \verb|condition| controllo da effettuare sui datasource interessati, le condition sono la sorgente degli allarmi generati
  \item \verb|timegraph| contengono informazioni sullo storico del determinato datasource, per esempio le statistiche di banda consumata per una data interfaccia di rete
\end{itemize}

% conditions
Le condition di un datagroup possono assumere 4 stati diversi:

\begin{figure}[H]
    \centering
    \includegraphics[height=0.6\linewidth]{build/condition_status.png}
    \caption{stati di una condition}
    \label{fig:enter-label}
\end{figure}

La transizione da uno stato all'altro avviene subito dopo l'esecuzione di una condition da parte di un poller, in particolare nel momento in cui l'espressione booleana della condition risulta falsa essa effettua una transizione nello stato \verb|failing|, al raggiungimento di un numero di tentativi falliti configurabile il controllo passa in stato \verb|down|, se il poller non riesce a valutare la condizione per esempio per irraggiungibilità del nodo o per fallimento di servizi necessari la condition transita in stato \verb|uncheckable|.

Il linguaggio di configurazione consente di specificare regole di dipendenza tra vari check per mezzo del parametro \verb|dependson|, una data condition che e in \verb|dependson| di un altra non viene verificata se le condition da cui dipende non si trovano in stato \verb|UP|, in caso questo non sia verificato la condition passa in stato \verb|uncheckable|

\begin{figure}[H]
    \centering
    \includegraphics[height=0.6\linewidth]{build/dependson_management.png}
    \caption{stati di una condition}
    \label{fig:enter-label}
\end{figure}

% tipologie di oggetti monitorabili
Sanet suddivide gli oggetti monitorabili in macro categorie note nel contesto del monitoraggio di apparati di rete, viene definita la seguente classificazione:

\begin{itemize}
  \item storage: supporti di memoria di vario genere sia fisici che virtuali
  \item service: servizi di qualunque entità, generalmente processi demoni
  \item interface: Interfacce di rete del nodo monitorato
\end{itemize}

Per questi oggetti vengono previsti datasource e condition di default come lo stato dello storage o il traffico sulle interfacce.

% generazione e instradamento di allarmi
Nel momento in cui una condition transita nello stato \verb|down| viene generato un allarme, questo viene processato dal demone \verb|entables| che determina la corretta destinazione dell'allarme in base a tabelle e catene di regole, in particolare sono previste 2 tabelle principali

\begin{itemize}
  \item tabella di mangle: utilizzata per arricchire l'allarme di metadati per il processing da parte di sistemi esterni
  \item tabella di filter: questa tabella e quella dove il pacchetto viene instradato al corretto destinatario
\end{itemize}

Le tabelle sono composte da catene di regole, nel momento in cui un allarme viene processato da \verb|entables| questo scorre tutte le catene definite e se una data regola nella catena fa match con gli attributi dell'allarme vengono applicate le azioni di quella determinata regola e si procede alla regola successiva

\begin{figure}[H]
    \centering
    \includegraphics[height=0.5\linewidth]{build/entables_processing.png}
    \caption{funzionamento di entables}
    \label{fig:enter-label}
\end{figure}

Un allarme può essere consegnato a destinazioni differenti, le più utilizzate sono la notifica attraverso \verb|SMTP| (email) oppure la consegna al sottosistema \verb|nocview|

% nocview comunicazione
Un allarme che fa match con la regola di instradamento \verb|store-alarms| viene destinato al sistema di notifica centrale dei laboratori: \verb|nocview|, la consegna dell'allarme avviene attraverso un modello di comunicazione polling, dove \verb|entables| salva l'allarme all'interno di un apposita tabella su database e \verb|nocview| esegue richieste http periodiche per recuperare gli allarmi della data tabella.
Gli allarmi raccolti dalle varie istanze vengono presentati ai membri del team NOC per mezzo di una singola interfaccia web suddivisi per istanza di sanet da cui sono stati estratti.

\begin{figure}[H]
    \centering
    \includegraphics[width=0.5\linewidth]{build/sanet_nocview_comunication.png}
    \caption{interazione sanet nocview}
    \label{fig:enter-label}
\end{figure}

% scheduling delle conditions
Le condition vengono verificate dal sistema applicando uno schema a polling, il demone centrale inserisce nella coda le condition che devono essere verificate e i processi poller consumano le condition in coda e inseriscono i risultati che vengono raccolti dal processo centrale che aggiorna lo stato interno delle conditions.Le code vengono implementate per mezzo del key value store redis

Durante l'esecuzione della condition il poller recupera i valori di tutti i datasource coinvolti nella condition stessa, esegue l'espressione booleana della condition, aggiorna i \verb|timegraph| con i rispettivi valori e restituisce tutto al demone centrale per mezzo delle code di redis. In caso il controllo preveda l'invio di un pacchetto ICMP per verificare la connettività viene coinvolto i demone \verb|pingerd| che si occupa di effettuare le comunicazioni di rete

\begin{figure}[H]
    \centering
    \includegraphics[width=0.5\linewidth]{build/sanetd_pollers_interaction.png}
    \caption{interazione sanetd pollers}
    \label{fig:enter-label}
\end{figure}

% linguaggio di configurazione
I nodi da monitorare e i rispettivi \verb|datagroup,storage,service,interface| vengono definiti per mezzo di un apposito linguaggio di configurazione, un esempio e il seguente

\begin{lstlisting}
node my.important.server
    agent main-agent
    description "Critical service for production, needs to be monitored"
    icon linux
    datagroup important datagroup
      condition important condition
      expr "2>1"
      exit
    exit
    storage rootfs distinguisher /
    exit
    interface ethernet distinguisher eth0
    exit
exit
\end{lstlisting}

Per poter distinguere particolari istanze di interfacce e storage si fa uso del parametro \verb|distinguisher| che viene utilizzato in fase di scansione della tabella SNMP che contiene informazioni sulle interfacce e sugli storage per identificare i dati interessati.

Il linguaggio di configurazione dei nodi consente inoltre di definire porzioni di configurazione cosiddetti \verb|templates| che consentono di raggruppare datagroup comuni per il monitoraggio di nodi in un unica configurazione e applicare una forma di ereditarietà, per esempio se si vogliono monitorare 10 macchine linux un possibile template e il seguente:


\begin{lstlisting}
node-template server-linux
    source site
    description "Server Linux"
    is-switch false
    is-router false
    datagroup clock-hr-tz
    exit
    datagroup cpu-hr
    exit
    datagroup defgw-mac-change
    exit
    datagroup icmp-reachability
    exit
    datagroup loadavg-threshold-ucd
    exit
    datagroup nonidle-ucd
    exit
    datagroup process-vlock-absence
    exit
    datagroup ram-usage-ucd
    exit
    datagroup reboot-hr
    exit
    datagroup snmp-status
    exit
    datagroup storage-all-linux
    exit
    datagroup swap-ucd
    exit
exit
\end{lstlisting}

Dove si definisce il monitoraggio dei servizi base di una macchina linux insieme alle componenti essenziali per le statistiche
I template possono contenere \verb|datagroups,condition,storages,interfaces,services|,essi sono a loro volta classificati in base alla risorsa di cui sono template:

\begin{itemize}
  \item \verb|node-templates| templates da applicare al nodo monitorato, possono contenere tutti gli altri template
  \item \verb|service-templates| templates per singolo servizio applicativo
  \item \verb|storage-templates| templates per lo storage
  \item \verb|interface-templates| templates per le interfacce
  \item \verb|datagroup-templates| templates per singolo datagroup, possono essere contenuti in tutti gli altri template e sono i maggiormente utilizzati
\end{itemize}

I template piu comuni come quello di cui sopra vengono mantenuti in una raccolta comune a tutte le installazioni detta \verb|library|, essa viene importata all'interno delle istanze di sanet e i template al suo interno vengono customizzati per mezzo di quanto necessario

\begin{figure}[H]
    \centering
    \includegraphics[width=1\linewidth]{build/sanet_library.png}
    \caption{sanet library}
    \label{fig:enter-label}
\end{figure}

% linguaggio di espressione delle condition
Le espressioni all'interno delle condition fanno uso di un altro linguaggio denominato \verb|sanet standard expression language|, pensato per la manipolazione dei valori di ritorno delle OID SNMP, i controlli fanno uso di funzioni implementate in tale linguaggio, se per esempio si vuole moniotrare la presenza di un servizio all'interno di un nodo la condition si mostrera come segue

\begin{lstlisting}
condition process_presence
  expr "(1.3.6.1.2.1.1.1.0@) != '' and isDefined('$swrunindex', catch_timeouts=False) == True"
exit
\end{lstlisting}

In ottica di estensibilità il linguaggio consente anche l'esecuzione di binari e script custom, per il monitoraggio di funzionalita non previste dalle funzioni builtin del linguaggio.

% tech stack di sviluppo TODO
Dal punto di vista di sviluppo sanet e formato da un insieme di moduli python che fanno uso estensivo del framework django per la gestione della webui e della persistenza a database,

% agenti remoti
Per monitorare i nodi di una data rete e necessario che i processi di sanet possano raggiungere gli apparati interessati, questo in produzione none sempre possibile per una serie di motivi quali partizionamento della rete per suddivisione dei compiti, regole firewall che non consentono il traffico dalla macchina di monitoraggio, inoltre in alcune installazioni geograficamente estese il monitoraggio risulta inefficiente in quanto le distanze rendono le misure di RTT e i timeout troppo stringenti, per affrontare il problema sanet offre la possibilita di utilizzare un poller remoto

\begin{figure}[H]
    \centering
    \includegraphics[width=0.5\linewidth]{build/remote_poller_architecture.png}
    \caption{architettura del poller remoto}
    \label{fig:enter-label}
\end{figure}

Il poller remoto e pensato come un servizio stateless che comunica con il demone sanetd per mezzo della istanza di redis del demone sanetd, ottiene la lista di condition da eseguire e restituisce al demone principale i risultati, notare che la comunicazione e iniziata dal poller remoto


\begin{figure}[H]
    \centering
    \includegraphics[width=0.5\linewidth]{build/remote_poller_sanetd_interaction.png}
    \caption{interazione sanetd poller remoto}
    \label{fig:enter-label}
\end{figure}

% sum up
Sanet si mostra come un complesso sistema distribuito in grado di adattarsi al contesto di produzione e fornire agli amministratori della rete un buon potere espressivo nel definire le opearazioni richieste per il monitoraggio

%\subsection{Analisi del contesto aziendale}
%
%Per ogni cliente dei Laboratori viene predisposta un istanza del sistema Sanet in una macchina dedicata presso l'infrastruttura IT del cliente, l'amministrazione e affidata ai sistemisti che hanno in carico la commessa dello specifico cliente
%
%Le richieste di messa a monitoraggio dei clienti vengono svolte dai membri del team o dai membri del Network Operation Center (NOC) a seconda della complessità dell'operazione.
%
%\begin{figure}[H]
%    \centering
%    \includegraphics[width=1\linewidth]{build/team_schema.png}
%    \caption{schema operativita team sistemistici}
%    \label{fig:enter-label}
%\end{figure}
%
%I clienti possono inoltre richiedere integrazioni ad hoc con software esterni, queste richieste vengono prese in carico dai team di riferimento.
%
%Con il tempo questa gestione ha portato a una frammentazione nelle scelte amministrative delle istanze di sanet e a gravi situazioni di software erosion. Ad ogni richiesta di monitoraggio dei clienti le installazioni di sanet vengono integrate con software esterni per mezzo di componenti sviluppati dai sistemisti.
%
%Queste componenti software non vengono sottoposte a versioning o documentazione estensiva e lo sviluppo non e in grado di tenerne traccia, inoltre non e previsto un controllo delle dipendenze ne una procedura di deployment.
%
%Non e in frequente, inoltre che queste integrazioni condividano dipendenze in comune con Sanet, queste possono mostrarsi sotto la forma di librerie ma anche di istanze di servizi software come redis o postgres
%
%Le integrazioni si interfacciano con Sanet per mezzo di API non standard che possono spaziare da formato di file generati dal sistema a output da command line interface.
%
%Questo approccio porta a un problema legato alle gestione delle manutenzioni del software, aggiornamento delle dipendenze e gestione delle manutenzioni ordinarie e straordinarie, per esempio l'aggiornamento dei sorgenti di sanet non può essere apportato in maniera identica su diverse installazioni senza rischiare di rompere integrazioni.
%
%Operazioni ordinarie come l'aggiunta a monitoraggio la modifica di configurazioni di sistema e la gestione di richieste del cliente non può essere svolta in maniera omogenea trasversalmente nelle diverse installazioni
%
%
%\chapter{Progettazione della soluzione}
%Progettazione della architettura di produzione del software applicativo, gestione di tutti i componenti critici per la corretta erogazione dei servizi offerti e progettazione del sistema di software management per la gestione del life-cycle, progettazione di tutte le operazioni necessarie per la migrazione alla nuova architettura
%
%\chapter{Implementazione}
%Implementazioni delle soluzioni progettate, descrizione delle problematiche incontrate in fase di migrazione e correzioni adottate per sopperire ai problemi
%
%\chapter{sviluppi futuri}
%Possibili sviluppi futuri sia in ambito architetturale sia per quanto riguarda il sistema di mantenimento dell'installato
%
%\chapter{Conclusioni}
%Considerazioni sulle performance, obbiettivi conseguiti,

\backmatter
\addcontentsline{toc}{chapter}{Bibliografia}
\bibliographystyle{unsrt}
% i riferimenti bibliografici, che devono essere almeno 20 per una tesi triennale ed almeno 30 per una della magistrale, si scaricano da qui https://dblp.uni-trier.de/search/
% e si aggiungono al file bibliografia.bib, dopodichè si citano opportuanamente nel testo della tesi con \cite{label}  dove label è il primo elemento di ogni rif. bibliografico subito dopo la parentesi graffa aperta, e.g. DBLP:books/daglib/0087929 (vedi file .bib sopra menzionato)
%\bibliography{bibliography}

\end{document}
