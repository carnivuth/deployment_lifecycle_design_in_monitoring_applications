\chapter{Analisi del problema di business}
%Ampia descrizione delle problematiche operative (workflow aziendali non più manutenibili) e di sviluppo (mantenimento dell'applicativo, automazione delle operazioni di ciclo di vita del software) overview di ciò che già e stato fatto e come viene gestito il software in produzione

\subsection{Analisi dell'applicativo}
% Analisi della situazione logistica aziendale, workflow di gestione di sanet, capacita operativa del team NOC in tale contesto
Il business principale dei laboratori Marconi consiste nel monitoraggio di apparati e servizi applicativi, per erogare tale servizio e non dipendere da software esterno l'azienda ha deciso di sviluppare la sua soluzione di monitoraggio proprietaria: \verb|sanet|.

Dal punto di vista architetturale Sanet e un sistema distribuito basato su Simple network message protocol (\verb|SNMP|), composto da processi demoni di varia natura tra cui:

\begin{itemize}
    \item demoni per lo storage dei dati
    \item demoni per il recupero dei dati stessi
    \item demoni per la gestione delle notifiche
\end{itemize}

\begin{figure}[H]
    \centering
    \includegraphics[width=1\linewidth]{build/sanet_architecture.png}
    \caption{architettura di sanet}
    \label{fig:enter-label}
\end{figure}

% architettura applicativa
L'architettura si compone di un demone principale sanetd che svolge i seguenti compiti:

\begin{itemize}
  \item gestione del flusso dati in ingresso e della ripartizione e scheduling dei task da eseguire per effettuare il monitoraggio
  \item salvataggio di log applicativi e stato dei controlli su storage stabile
  \item gestione delle timelines generate dai task di monitoraggio
  \item gestione del flusso in ingresso per il demone di gestione delle notifiche
\end{itemize}

% datagroups
I task generati dal demone principale vengono denominati \verb|datagroups|, un datagroup e la composizione di tre elementi principali:

\begin{itemize}
  \item \verb|datasource| sorgente dei dati, specifica le modalità con cui i dati devono essere reperiti, per esempio possono essere specificate OID SNMP, files, risorse web
  \item \verb|condition| controllo da effettuare sui datasource interessati, le condition sono la sorgente degli allarmi generati
  \item \verb|timegraph| contengono informazioni sullo storico del determinato datasource, per esempio le statistiche di banda consumata per una data interfaccia di rete
\end{itemize}

% conditions
Le condition di un datagroup possono assumere 4 stati diversi:

\begin{figure}[H]
    \centering
    \includegraphics[height=0.6\linewidth]{build/condition_status.png}
    \caption{stati di una condition}
    \label{fig:enter-label}
\end{figure}

La transizione da uno stato all'altro avviene subito dopo l'esecuzione di una condition da parte di un poller, in particolare nel momento in cui l'espressione booleana della condition risulta falsa essa effettua una transizione nello stato \verb|failing|, al raggiungimento di un numero di tentativi falliti configurabile il controllo passa in stato \verb|down|, se il poller non riesce a valutare la condizione per esempio per irraggiungibilità del nodo o per fallimento di servizi necessari la condition transita in stato \verb|uncheckable|.

Il linguaggio di configurazione consente di specificare regole di dipendenza tra vari check per mezzo del parametro \verb|dependson|, una data condition che e in \verb|dependson| di un altra non viene verificata se le condition da cui dipende non si trovano in stato \verb|UP|, in caso questo non sia verificato la condition passa in stato \verb|uncheckable|

\begin{figure}[H]
    \centering
    \includegraphics[height=0.6\linewidth]{build/dependson_management.png}
    \caption{gestione di condition in dependson}
    \label{fig:enter-label}
\end{figure}

% tipologie di oggetti monitorabili
Sanet suddivide gli oggetti da sottoporre al monitoraggio in macro categorie note nel contesto del monitoraggio di apparati di rete, viene definita la seguente classificazione:

\begin{itemize}
  \item storage: supporti di memoria di vario genere sia fisici che virtuali
  \item service: servizi di qualunque entità, generalmente processi demoni
  \item interface: Interfacce di rete del nodo monitorato
\end{itemize}

Per questi oggetti vengono previsti datasource e condition di default come lo stato dello storage o il traffico sulle interfacce.

% generazione e instradamento di allarmi
Nel momento in cui una condition transita nello stato \verb|down| viene generato un allarme, questo viene processato dal demone \verb|entables| che determina la corretta destinazione dell'allarme in base a tabelle e catene di regole, in particolare sono previste 2 tabelle principali

\begin{itemize}
  \item tabella di mangle: utilizzata per arricchire l'allarme di metadati per il processing da parte di sistemi esterni
  \item tabella di filter: questa tabella e quella dove il pacchetto viene instradato al corretto destinatario
\end{itemize}

Le tabelle sono composte da catene di regole, nel momento in cui un allarme viene processato da \verb|entables| questo scorre tutte le catene definite e se una data regola nella catena fa match con gli attributi dell'allarme vengono applicate le azioni di quella determinata regola e si procede alla regola successiva

\begin{figure}[H]
    \centering
    \includegraphics[height=0.6\linewidth]{build/entables_processing.png}
    \caption{funzionamento di entables}
    \label{fig:enter-label}
\end{figure}

Un allarme può essere consegnato a destinazioni differenti, le più utilizzate sono la notifica attraverso \verb|SMTP| (email) oppure la consegna al sottosistema \verb|nocview|

% nocview comunicazione
Un allarme che fa match con la regola di instradamento \verb|store-alarms| viene destinato al sistema di notifica centrale dei laboratori: \verb|nocview|, la consegna dell'allarme avviene attraverso un modello di comunicazione polling, dove \verb|entables| salva l'allarme all'interno di un apposita tabella su database e \verb|nocview| esegue richieste http periodiche per recuperare gli allarmi della data tabella.
Gli allarmi raccolti dalle varie istanze vengono presentati ai membri del team NOC per mezzo di una singola interfaccia web suddivisi per istanza di sanet da cui sono stati estratti.

\begin{figure}[H]
    \centering
    \includegraphics[width=0.6\linewidth]{build/sanet_nocview_comunication.png}
    \caption{interazione sanet nocview}
    \label{fig:enter-label}
\end{figure}

% scheduling delle conditions
Le condition vengono verificate dal sistema applicando uno schema a polling, il demone centrale inserisce nella coda le condition che devono essere verificate e i processi poller consumano le condition in coda e inseriscono i risultati che vengono raccolti dal processo centrale che aggiorna lo stato interno delle conditions.Le code vengono implementate per mezzo del key value store redis

Durante l'esecuzione della condition il poller recupera i valori di tutti i datasource coinvolti nella condition stessa, esegue l'espressione booleana della condition, aggiorna i \verb|timegraph| con i rispettivi valori e restituisce tutto al demone centrale per mezzo delle code di redis. In caso il controllo preveda l'invio di un pacchetto ICMP per verificare la connettività viene coinvolto i demone \verb|pingerd| che si occupa di effettuare le comunicazioni di rete

\begin{figure}[H]
    \centering
    \includegraphics[width=0.6\linewidth]{build/sanetd_pollers_interaction.png}
    \caption{Interazione sanetd pollers}
    \label{fig:enter-label}
\end{figure}

% linguaggio di configurazione
I nodi da monitorare e i rispettivi \verb|datagroup,storage,service,interface| vengono definiti per mezzo di un apposito linguaggio di configurazione, un esempio e il seguente

\begin{lstlisting}
node my.important.server
    agent main-agent
    description "Critical service for production, needs to be monitored"
    icon linux
    datagroup important datagroup
      condition important condition
      expr "2>1"
      exit
    exit
    storage rootfs distinguisher /
    exit
    interface ethernet distinguisher eth0
    exit
exit
\end{lstlisting}

Per poter distinguere particolari istanze di interfacce e storage si fa uso del parametro \verb|distinguisher| che viene utilizzato in fase di scansione della tabella SNMP che contiene informazioni sulle interfacce e sugli storage per identificare i dati interessati.

Il linguaggio di configurazione dei nodi consente inoltre di definire porzioni di configurazione cosiddetti \verb|templates| che consentono di raggruppare datagroup comuni per il monitoraggio di nodi in un unica configurazione e applicare una forma di ereditarietà, per esempio se si vogliono monitorare 10 macchine linux un possibile template e il seguente:


\begin{lstlisting}
node-template server-linux
    source site
    description "Server Linux"
    is-switch false
    is-router false
    datagroup clock-hr-tz
    exit
    datagroup cpu-hr
    exit
    datagroup defgw-mac-change
    exit
    datagroup icmp-reachability
    exit
    datagroup loadavg-threshold-ucd
    exit
    datagroup nonidle-ucd
    exit
    datagroup process-vlock-absence
    exit
    datagroup ram-usage-ucd
    exit
    datagroup reboot-hr
    exit
    datagroup snmp-status
    exit
    datagroup storage-all-linux
    exit
    datagroup swap-ucd
    exit
exit
\end{lstlisting}

Dove si definisce il monitoraggio dei servizi base di una macchina linux insieme alle componenti essenziali per le statistiche
I template possono contenere \verb|datagroups,condition,storages,interfaces,services|,essi sono a loro volta classificati in base alla risorsa di cui sono template:

\begin{itemize}
  \item \verb|node-templates| templates da applicare al nodo monitorato, possono contenere tutti gli altri template
  \item \verb|service-templates| templates per singolo servizio applicativo
  \item \verb|storage-templates| templates per lo storage
  \item \verb|interface-templates| templates per le interfacce
  \item \verb|datagroup-templates| templates per singolo datagroup, possono essere contenuti in tutti gli altri template e sono i maggiormente utilizzati
\end{itemize}

I template più comuni come quello di cui sopra vengono mantenuti in una raccolta comune a tutte le installazioni detta \verb|library|, essa viene importata all'interno delle istanze di sanet e i template al suo interno vengono personalizzati per mezzo di quanto necessario

\begin{figure}[H]
    \centering
    \includegraphics[width=1\linewidth]{build/sanet_library.png}
    \caption{sanet library}
    \label{fig:enter-label}
\end{figure}

% linguaggio di espressione delle condition
Le espressioni all'interno delle condition fanno uso di un altro linguaggio denominato \verb|sanet standard expression language|, pensato per la manipolazione dei valori di ritorno delle OID SNMP, i controlli fanno uso di funzioni implementate in tale linguaggio, se per esempio si vuole monitorare la presenza di un servizio all'interno di un nodo la condition si mostrerà come segue

\begin{lstlisting}
condition process_presence
  expr "(1.3.6.1.2.1.1.1.0@) != '' and isDefined('\$swrunindex', catch_timeouts=False) == True"
exit
\end{lstlisting}

In ottica di estensibilità il linguaggio consente anche l'esecuzione di binari e script custom, per il monitoraggio di funzionalità non previste dalle funzioni builtin del linguaggio.

% tech stack di sviluppo TODO
Dal punto di vista di sviluppo sanet e formato da un insieme di moduli python che fanno uso estensivo del framework django per la gestione della webui e della persistenza a database,

% agenti remoti
Per monitorare i nodi di una data rete e necessario che i processi di sanet possano raggiungere gli apparati interessati, questo in produzione none sempre possibile per una serie di motivi quali partizionamento della rete per suddivisione dei compiti, regole firewall che non consentono il traffico dalla macchina di monitoraggio, inoltre in alcune installazioni geograficamente estese il monitoraggio risulta inefficiente in quanto le distanze rendono le misure di RTT esagerate e i timeout troppo stringenti, per affrontare il problema sanet offre la possibilità di utilizzare un poller remoto

\begin{figure}[H]
    \centering
    \includegraphics[width=0.3\linewidth]{build/remote_poller_architecture.png}
    \caption{architettura del poller remoto}
    \label{fig:enter-label} \end{figure}

Il poller remoto e pensato come un servizio stateless che comunica con il demone sanetd per mezzo della istanza di redis del demone sanetd, ottiene la lista di condition da eseguire e restituisce al demone principale i risultati, notare che la comunicazione e iniziata dal poller remoto


\begin{figure}[H]
    \centering
    \includegraphics[width=0.8\linewidth]{build/remote_poller_sanetd_interaction.png}
    \caption{interazione sanetd poller remoto}
    \label{fig:enter-label}
\end{figure}

Sanet modella le entità che eseguono le condition come \verb|agenti|, ogni agente e affidato a un poller e ogni poller corrisponde a un set di processi pesanti unix e un sottoinsieme di thread python dedicati al particolare agente:


\begin{figure}[H]
    \centering
    \includegraphics[width=0.8\linewidth]{build/agents_vs_pollers.png}
    \caption{relazione tra poller e agenti}
    \label{fig:enter-label}
\end{figure}

% sum up
Sanet si mostra come un complesso sistema distribuito in grado di adattarsi al contesto di produzione e fornire agli amministratori della rete un buon potere espressivo nel definire le operazioni richieste per il monitoraggio

\subsection{Analisi del contesto aziendale}

% struttura area NMS
L'area network management and security (NMS) dei Laboratori e strutturata in Team, ogni team fa capo a un sistemista senior e risponde di un sottoinsieme di clienti. Per ogni cliente dei Laboratori viene predisposta un istanza del sistema Sanet in una macchina dedicata presso l'infrastruttura IT del cliente, l'amministrazione e affidata ai sistemisti che ne hanno in carico la commessa.

% gestione richieste di monitoraggio
Le richieste di messa a monitoraggio dei clienti vengono svolte dai membri del team o dai membri del Network Operation Center (NOC) a seconda della complessità dell'operazione.

\begin{figure}[H]
    \centering
    \includegraphics[height=0.6\linewidth]{build/team_schema.png}
    \caption{schema operatività team sistemistici}
    \label{fig:enter-label}
\end{figure}

Le richieste di monitoraggio possono spaziare tra una vasta gamma di casistiche tra cui:

\begin{itemize}
  \item{aggiunta a monitoraggio di elementi nuovi elementi noti}
  \item{monitoraggio di applicativi di varia natura (database,servizi web, posta)}
  \item{monitoraggio di infrastrutture hardware non note}
  \item{monitoraggio di nodi mobili}
\end{itemize}

In caso la richiesta preveda dello sviluppo software essa viene presa in carico dai membri del team che valutano e progettano un componente software che, interfacciandosi con sanet fornisce il servizio richiesto.

% workflow di update
I membri del team essi sono anche responsabili per la manutenzione e aggiornamento delle istanze di cui sono amministratori, il workflow di update prevede che l'area NAD notifichi il team della necessita di un aggiornamento delle istanze in produzione


\begin{figure}[H]
    \centering
    \includegraphics[width=1.2\linewidth]{build/update_workflow.png}
    \caption{procedura di update}
    \label{fig:enter-label}
\end{figure}

Il team inoltre determina come l'architettura di sanet venga distribuita sull'infrastruttura del cliente, considerando le risorse che il cliente mette a disposizione per la sua infrastruttura.

\subsection{Analisi stato della produzione}

% "artefatti" prodotti dal area NAD (stendiamo un velo pietoso....)
L'area NAD fornisce sanet sotto forma di repository subversion contenenti i sorgenti con una guida all'installazione e aggiornamento del software, l'applicativo e pensato per essere eseguito all'interno di un virtual environment di python dove vengono installati i moduli necessari, il sistema a runtime deve anche rendere disponibili i servizi attivi necessari al funzionamento come redis e postgres. Questi requisiti vengono soddisfatti dalla infrastruttura a runtime fornita dal team,

\begin{figure}[H]
    \centering
    \includegraphics[width=1\linewidth]{build/sanet_production_simple.png}
    \caption{architettura sanet in produzione}
    \label{fig:enter-label}
\end{figure}

% sanet e bind locale
Il sistema sanet, data la necessita di effettuare richieste di rete in maniera periodica genera un flusso costante di richieste DNS, questo porta a una degradazione delle prestazioni della rete e a statistiche di monitoraggio viziate come valori di RTT più alti e traffico in uscita dagli apparati di rete aumentato. Per migliorare la situazione e rendere il sistema più fault tolerant si introduce un server DNS locale al processo poller, questo si comporta da slave del DNS master del cliente, scarica le zone all'avvio e effettua la risoluzione dei nomi localmente al sistema sanet azzerando il traffico di rete.


\begin{figure}[H]
    \centering
    \includegraphics[width=1\linewidth]{build/sanet_bind.png}
    \caption{sanet e bind locale}
    \label{fig:enter-label}
\end{figure}

In questo modo sanet e in grado di continuare a erogare il servizio anche in caso di guasti al servizio DNS.

% sanet e le estensioni per il monitoraggio
Le integrazioni per monitoraggi si presentano sotto molte forme, quelle maggiormente riscontrate in produzione sono:

\begin{itemize}
  \item{script bash o python eseguiti direttamente dal poller}
  \item{script bash o python eseguiti periodicamente per mezzo del demone cron}
\end{itemize}

In caso di esecuzione per mezzo di cron l'integrazione e sanet interagiscono in maniera disaccoppiata nel tempo e nello spazio, lo script comunica con sanet per mezzo del filesystem, scrivendo file all'interno della macchina di monitoraggio o per mezzo del demone redis, aggiungendo chiavi al database

\begin{figure}[H]
    \centering
    \includegraphics[width=1\linewidth]{build/sanet_integrations.png}
    \caption{sanet e integrazioni sistemistiche}
    \label{fig:enter-label}
\end{figure}

Le competenze delle integrazioni possono variare da controlli infrastrutturali che monitorano lo stato di particolari servizi come tunnel vpn, backup job di virtualizzatori, stato di tablespaces oracle oppure monitorare particolari API di sistemi complessi come VMware.

% sanet e rancid
Una delle principali funzionalità che offre sanet e quella di monitorare apparati di rete che espongono agenti SNMP, uno dei compiti assegnati alla macchina di monitoraggio dai sistemisti e anche quello di eseguire il backup periodico degli apparati per mezzo del tool \verb|really awesome cisco config differ| (\verb|rancid|).

Il programma rancid sfrutta il programma \verb|clogin| per connettersi agli apparati, esegue dunque uno script \verb|perl| per recuperare le informazioni di configurazione dell'apparato in forma testuale e ne salva le modifiche per mezzo di un sistema di controllo di versione.

Per evitare di configurare rancid manualmente i sistemisti hanno sviluppato un integrazione con sanet che data una struttura dati che fa uso dei tag di sanet per recuperare dinamicamente la lista di nodi che devono essere sottoposti alla procedura di backup di rancid

\begin{figure}[H]
    \centering
    \includegraphics[width=1\linewidth]{build/sanet_rancid.png}
    \caption{Integrazione sanet rancid}
    \label{fig:enter-label}
\end{figure}

% sanet di backup
Per garantire il servizio anche in caso di guasti fisici all'infrastruttura che ospita il sistema in alcune installazioni grandi di sanet viene predisposta una seconda istanza parallela, questa esegue in maniera parallela alla precedente e il database di sanet viene mantenuto sincronizzato con la produzione principale, in caso di guasti l'istanza secondaria sostituisce la principale nell'invio delle notifiche al sistema \verb|nocview|

\begin{figure}[H]
    \centering
    \includegraphics[width=0.6\linewidth]{build/sanet_backup.png}
    \caption{Ridondanza nelle installazioni di sanet}
    \label{fig:enter-label}
\end{figure}

\subsection{analisi delle problematiche insorte}

% problematiche insorte
Delegare l'ingegnerizzazione delle istanze ai singoli team ha si ridotto il carico di lavoro dell'area NAD ma ha anche comportato con il tempo una situazione di frammentazione delle installazioni di sanet, le feature necessarie al monitoraggio sono implementate in maniera diversa a seconda della conoscenza a disposizione dei team di sanet e delle tecnologie utilizzabili, inoltre le informazioni.

Molte delle richieste di messa a monitoraggio anche se simili sono implementate in maniera diversa tra le diverse istanze di sanet, un esempio lampante e il monitoraggio di cluster VMware che e implementato per mezzo di integrazioni di sanet diverse fra le varie installazioni dei clienti.

Questo comporta un alto costo nel caso di lavori su istanze non note, un sistemista che deve eseguire operazioni di manutenzione e utilizzo di un sanet non noto può ritrovarsi incapace di agire efficacemente





Con il tempo questa gestione ha portato a una frammeinoltre ntazione nelle scelte amministrative delle istanze di sanet e a gravi situazioni di software erosion. Ad ogni richiesta di monitoraggio dei clienti le installazioni di sanet vengono integrate con software esterni per mezzo di componenti sviluppati dai sistemisti.
%
%Queste componenti software non vengono sottoposte a versioning o documentazione estensiva e lo sviluppo non e in grado di tenerne traccia, inoltre non e previsto un controllo delle dipendenze ne una procedura di deployment.
%
%Non e in frequente, inoltre che queste integrazioni condividano dipendenze in comune con Sanet, queste possono mostrarsi sotto la forma di librerie ma anche di istanze di servizi software come redis o postgres
%
%Le integrazioni si interfacciano con Sanet per mezzo di API non standard che possono spaziare da formato di file generati dal sistema a output da command line interface.
%
%Questo approccio porta a un problema legato alle gestione delle manutenzioni del software, aggiornamento delle dipendenze e gestione delle manutenzioni ordinarie e straordinarie, per esempio l'aggiornamento dei sorgenti di sanet non può essere apportato in maniera identica su diverse installazioni senza rischiare di rompere integrazioni.
%
%Operazioni ordinarie come l'aggiunta a monitoraggio la modifica di configurazioni di sistema e la gestione di richieste del cliente non può essere svolta in maniera omogenea trasversalmente nelle diverse installazioni
