\chapter{Analisi dei requisiti}
%Requisiti di progetto e obbiettivo finale richiesto dalla missione di business

L'attività svolta presso i laboratori Marconi aveva come obbiettivo finale rendere i membri del network operation center team (NOC in breve) capaci di svolgere le operazioni di manutenzione delle installazioni del software di monitoraggio aziendale Sanet, in particolare:

\begin{itemize}
\item svolgere in autonomia operazioni di aggiornamento distribuito di sanet
\item programmare e organizzare attività di installazione di istanze di sanet
\item aggiungere oggetti al processo di monitoraggio
\item svolgere monitoraggio delle performance e conseguenti attività di manutenzione come pulizia di log applicativi, bilanciamenti di carico, aggiunta di nodi al cluster di monitoraggio
\end{itemize}

Questo con l'obbiettivo di poter successivamente spostare la responsabilità della gestione delle istanze in produzione al team NOC, e evitare che i membri dei singoli team di sistemisti si debbano occupare di tale attività, liberando risorse da dedicare a task sistemistici richiesti dai clienti.

E' stato inoltre richiesto che fossero revisionate le documentazioni tecniche riguardanti le attività di monitoraggio svolte presso i vari clienti, in modo da uniformare la base documentale e renderne più semplice l'accesso in maniera programmatica.

Uno dei requisti fondamentali richiesti è quello di uniformare lo stato dell'installato di sanet che per via della gestione affidata a diversi team, verte in condizioni di estrema frammentazione dove configurazioni di monitoraggio, regole di accesso e linee guida per l'amministrazione sono diverse

I requisti richiesti hanno come scopo ultimo quello di ridurre la quantità di ore uomo complessive spese in operazioni di manutenzione di Sanet, ridurre l'entropia delle installazioni in produzione accentrandone la gestione a un unico team

